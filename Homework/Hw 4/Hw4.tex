\documentclass[]{article}
\usepackage{lmodern}
\usepackage{amssymb,amsmath}
\usepackage{ifxetex,ifluatex}
\usepackage{fixltx2e} % provides \textsubscript
\ifnum 0\ifxetex 1\fi\ifluatex 1\fi=0 % if pdftex
  \usepackage[T1]{fontenc}
  \usepackage[utf8]{inputenc}
\else % if luatex or xelatex
  \ifxetex
    \usepackage{mathspec}
  \else
    \usepackage{fontspec}
  \fi
  \defaultfontfeatures{Ligatures=TeX,Scale=MatchLowercase}
\fi
% use upquote if available, for straight quotes in verbatim environments
\IfFileExists{upquote.sty}{\usepackage{upquote}}{}
% use microtype if available
\IfFileExists{microtype.sty}{%
\usepackage{microtype}
\UseMicrotypeSet[protrusion]{basicmath} % disable protrusion for tt fonts
}{}
\usepackage[margin=1in]{geometry}
\usepackage{hyperref}
\hypersetup{unicode=true,
            pdftitle={Homework 4},
            pdfauthor={Xiang Yang Ng},
            pdfborder={0 0 0},
            breaklinks=true}
\urlstyle{same}  % don't use monospace font for urls
\usepackage{color}
\usepackage{fancyvrb}
\newcommand{\VerbBar}{|}
\newcommand{\VERB}{\Verb[commandchars=\\\{\}]}
\DefineVerbatimEnvironment{Highlighting}{Verbatim}{commandchars=\\\{\}}
% Add ',fontsize=\small' for more characters per line
\usepackage{framed}
\definecolor{shadecolor}{RGB}{248,248,248}
\newenvironment{Shaded}{\begin{snugshade}}{\end{snugshade}}
\newcommand{\KeywordTok}[1]{\textcolor[rgb]{0.13,0.29,0.53}{\textbf{#1}}}
\newcommand{\DataTypeTok}[1]{\textcolor[rgb]{0.13,0.29,0.53}{#1}}
\newcommand{\DecValTok}[1]{\textcolor[rgb]{0.00,0.00,0.81}{#1}}
\newcommand{\BaseNTok}[1]{\textcolor[rgb]{0.00,0.00,0.81}{#1}}
\newcommand{\FloatTok}[1]{\textcolor[rgb]{0.00,0.00,0.81}{#1}}
\newcommand{\ConstantTok}[1]{\textcolor[rgb]{0.00,0.00,0.00}{#1}}
\newcommand{\CharTok}[1]{\textcolor[rgb]{0.31,0.60,0.02}{#1}}
\newcommand{\SpecialCharTok}[1]{\textcolor[rgb]{0.00,0.00,0.00}{#1}}
\newcommand{\StringTok}[1]{\textcolor[rgb]{0.31,0.60,0.02}{#1}}
\newcommand{\VerbatimStringTok}[1]{\textcolor[rgb]{0.31,0.60,0.02}{#1}}
\newcommand{\SpecialStringTok}[1]{\textcolor[rgb]{0.31,0.60,0.02}{#1}}
\newcommand{\ImportTok}[1]{#1}
\newcommand{\CommentTok}[1]{\textcolor[rgb]{0.56,0.35,0.01}{\textit{#1}}}
\newcommand{\DocumentationTok}[1]{\textcolor[rgb]{0.56,0.35,0.01}{\textbf{\textit{#1}}}}
\newcommand{\AnnotationTok}[1]{\textcolor[rgb]{0.56,0.35,0.01}{\textbf{\textit{#1}}}}
\newcommand{\CommentVarTok}[1]{\textcolor[rgb]{0.56,0.35,0.01}{\textbf{\textit{#1}}}}
\newcommand{\OtherTok}[1]{\textcolor[rgb]{0.56,0.35,0.01}{#1}}
\newcommand{\FunctionTok}[1]{\textcolor[rgb]{0.00,0.00,0.00}{#1}}
\newcommand{\VariableTok}[1]{\textcolor[rgb]{0.00,0.00,0.00}{#1}}
\newcommand{\ControlFlowTok}[1]{\textcolor[rgb]{0.13,0.29,0.53}{\textbf{#1}}}
\newcommand{\OperatorTok}[1]{\textcolor[rgb]{0.81,0.36,0.00}{\textbf{#1}}}
\newcommand{\BuiltInTok}[1]{#1}
\newcommand{\ExtensionTok}[1]{#1}
\newcommand{\PreprocessorTok}[1]{\textcolor[rgb]{0.56,0.35,0.01}{\textit{#1}}}
\newcommand{\AttributeTok}[1]{\textcolor[rgb]{0.77,0.63,0.00}{#1}}
\newcommand{\RegionMarkerTok}[1]{#1}
\newcommand{\InformationTok}[1]{\textcolor[rgb]{0.56,0.35,0.01}{\textbf{\textit{#1}}}}
\newcommand{\WarningTok}[1]{\textcolor[rgb]{0.56,0.35,0.01}{\textbf{\textit{#1}}}}
\newcommand{\AlertTok}[1]{\textcolor[rgb]{0.94,0.16,0.16}{#1}}
\newcommand{\ErrorTok}[1]{\textcolor[rgb]{0.64,0.00,0.00}{\textbf{#1}}}
\newcommand{\NormalTok}[1]{#1}
\usepackage{graphicx,grffile}
\makeatletter
\def\maxwidth{\ifdim\Gin@nat@width>\linewidth\linewidth\else\Gin@nat@width\fi}
\def\maxheight{\ifdim\Gin@nat@height>\textheight\textheight\else\Gin@nat@height\fi}
\makeatother
% Scale images if necessary, so that they will not overflow the page
% margins by default, and it is still possible to overwrite the defaults
% using explicit options in \includegraphics[width, height, ...]{}
\setkeys{Gin}{width=\maxwidth,height=\maxheight,keepaspectratio}
\IfFileExists{parskip.sty}{%
\usepackage{parskip}
}{% else
\setlength{\parindent}{0pt}
\setlength{\parskip}{6pt plus 2pt minus 1pt}
}
\setlength{\emergencystretch}{3em}  % prevent overfull lines
\providecommand{\tightlist}{%
  \setlength{\itemsep}{0pt}\setlength{\parskip}{0pt}}
\setcounter{secnumdepth}{0}
% Redefines (sub)paragraphs to behave more like sections
\ifx\paragraph\undefined\else
\let\oldparagraph\paragraph
\renewcommand{\paragraph}[1]{\oldparagraph{#1}\mbox{}}
\fi
\ifx\subparagraph\undefined\else
\let\oldsubparagraph\subparagraph
\renewcommand{\subparagraph}[1]{\oldsubparagraph{#1}\mbox{}}
\fi

%%% Use protect on footnotes to avoid problems with footnotes in titles
\let\rmarkdownfootnote\footnote%
\def\footnote{\protect\rmarkdownfootnote}

%%% Change title format to be more compact
\usepackage{titling}

% Create subtitle command for use in maketitle
\newcommand{\subtitle}[1]{
  \posttitle{
    \begin{center}\large#1\end{center}
    }
}

\setlength{\droptitle}{-2em}
  \title{Homework 4}
  \pretitle{\vspace{\droptitle}\centering\huge}
  \posttitle{\par}
  \author{Xiang Yang Ng}
  \preauthor{\centering\large\emph}
  \postauthor{\par}
  \predate{\centering\large\emph}
  \postdate{\par}
  \date{May 17, 2018}


\begin{document}
\maketitle

\subsection{Problem 1}\label{problem-1}

Download all the libraries and set the directory

\begin{Shaded}
\begin{Highlighting}[]
\KeywordTok{rm}\NormalTok{(}\DataTypeTok{list=}\KeywordTok{ls}\NormalTok{(}\DataTypeTok{all=}\NormalTok{T))}
\KeywordTok{setwd}\NormalTok{(}\StringTok{"/Users/Xiang/OneDrive/Desktop/Econ-144/Homework/Hw 4"}\NormalTok{)}
\KeywordTok{library}\NormalTok{(forecast)}
\KeywordTok{library}\NormalTok{(stats)}
\KeywordTok{library}\NormalTok{(timeSeries)}
\end{Highlighting}
\end{Shaded}

\begin{verbatim}
## Loading required package: timeDate
\end{verbatim}

\begin{Shaded}
\begin{Highlighting}[]
\KeywordTok{library}\NormalTok{(tseries)}
\end{Highlighting}
\end{Shaded}

Read the dataset

\begin{Shaded}
\begin{Highlighting}[]
\NormalTok{data =}\StringTok{ }\KeywordTok{read.table}\NormalTok{ (}\StringTok{"w-gs1yr.txt"}\NormalTok{,}\DataTypeTok{header =} \OtherTok{TRUE}\NormalTok{)}
\KeywordTok{attach}\NormalTok{(data)}
\end{Highlighting}
\end{Shaded}

\begin{enumerate}
\def\labelenumi{\alph{enumi})}
\item
\end{enumerate}

\begin{Shaded}
\begin{Highlighting}[]
\CommentTok{#create a time series object for the data}
\NormalTok{ir_ts =}\StringTok{ }\KeywordTok{ts}\NormalTok{(rate, }\DataTypeTok{start =} \DecValTok{1962}\OperatorTok{+}\NormalTok{(}\DecValTok{5}\OperatorTok{/}\DecValTok{365}\NormalTok{), }\DataTypeTok{freq =} \DecValTok{52}\NormalTok{)}

\CommentTok{#display the plot, acf and pacf}
\KeywordTok{tsdisplay}\NormalTok{(ir_ts, }\DataTypeTok{main =} \StringTok{"Plot of weekly interest rate and its ACF and PACF"}\NormalTok{)}
\end{Highlighting}
\end{Shaded}

\includegraphics{Hw4_files/figure-latex/unnamed-chunk-3-1.pdf}

There is very strong persistence in the plot, which might mean that
there is cycles in the data. There doesn't seem to have to have a strong
upward or downward trend. Looking at the PACF, we can see that there is
presence of seasonality, particularly S-MA(3) model might fit the data.
Two strong spikes in the PACF might also suggest an AR(2) process.

\begin{enumerate}
\def\labelenumi{\alph{enumi})}
\setcounter{enumi}{1}
\item
\end{enumerate}

\begin{Shaded}
\begin{Highlighting}[]
\CommentTok{#Constructing the models}
\NormalTok{arma1 =}\StringTok{ }\KeywordTok{Arima}\NormalTok{(ir_ts, }\DataTypeTok{order =} \KeywordTok{c}\NormalTok{(}\DecValTok{2}\NormalTok{,}\DecValTok{0}\NormalTok{,}\DecValTok{0}\NormalTok{), }\DataTypeTok{seasonal =} \KeywordTok{list}\NormalTok{(}\DataTypeTok{order =} \KeywordTok{c}\NormalTok{(}\DecValTok{0}\NormalTok{,}\DecValTok{0}\NormalTok{,}\DecValTok{3}\NormalTok{), }\DataTypeTok{period =} \DecValTok{6}\NormalTok{))}
\NormalTok{arma2 =}\StringTok{ }\KeywordTok{Arima}\NormalTok{(ir_ts, }\DataTypeTok{order =} \KeywordTok{c}\NormalTok{(}\DecValTok{3}\NormalTok{,}\DecValTok{0}\NormalTok{,}\DecValTok{0}\NormalTok{), }\DataTypeTok{seasonal =} \KeywordTok{list}\NormalTok{(}\DataTypeTok{order =} \KeywordTok{c}\NormalTok{(}\DecValTok{0}\NormalTok{,}\DecValTok{0}\NormalTok{,}\DecValTok{3}\NormalTok{), }\DataTypeTok{period =} \DecValTok{5}\NormalTok{))}
\NormalTok{arma3 =}\StringTok{ }\KeywordTok{Arima}\NormalTok{(ir_ts, }\DataTypeTok{order =} \KeywordTok{c}\NormalTok{(}\DecValTok{3}\NormalTok{,}\DecValTok{0}\NormalTok{,}\DecValTok{0}\NormalTok{), }\DataTypeTok{seasonal =} \KeywordTok{list}\NormalTok{(}\DataTypeTok{order =} \KeywordTok{c}\NormalTok{(}\DecValTok{0}\NormalTok{,}\DecValTok{0}\NormalTok{,}\DecValTok{2}\NormalTok{), }\DataTypeTok{period =} \DecValTok{6}\NormalTok{))}

\KeywordTok{plot}\NormalTok{(arma1}\OperatorTok{$}\NormalTok{x,}\DataTypeTok{col=}\StringTok{"red"}\NormalTok{, }\DataTypeTok{main =} \StringTok{"Plot of the first model and the time series"}\NormalTok{, }\DataTypeTok{xlab =} \StringTok{"Time"}\NormalTok{, }\DataTypeTok{ylab =} \StringTok{"Predicted and actual values"}\NormalTok{)}
\KeywordTok{legend}\NormalTok{(}\StringTok{"topright"}\NormalTok{, }\DataTypeTok{legend =} \KeywordTok{c}\NormalTok{(}\StringTok{"Predicted"}\NormalTok{, }\StringTok{"Actual"}\NormalTok{), }\DataTypeTok{lty =} \DecValTok{1}\NormalTok{, }\DataTypeTok{col =} \KeywordTok{c}\NormalTok{(}\StringTok{"blue"}\NormalTok{,}\StringTok{"red"}\NormalTok{))}
\KeywordTok{lines}\NormalTok{(}\KeywordTok{fitted}\NormalTok{(arma1),}\DataTypeTok{col=}\StringTok{"blue"}\NormalTok{)}
\end{Highlighting}
\end{Shaded}

\includegraphics{Hw4_files/figure-latex/unnamed-chunk-4-1.pdf}

\begin{Shaded}
\begin{Highlighting}[]
\KeywordTok{plot}\NormalTok{(arma2}\OperatorTok{$}\NormalTok{x,}\DataTypeTok{col=}\StringTok{"red"}\NormalTok{, }\DataTypeTok{main =} \StringTok{"Plot of the second model and the time series"}\NormalTok{, }\DataTypeTok{xlab =} \StringTok{"Time"}\NormalTok{, }\DataTypeTok{ylab =} \StringTok{"Predicted and actual values"}\NormalTok{)}
\KeywordTok{legend}\NormalTok{(}\StringTok{"topright"}\NormalTok{, }\DataTypeTok{legend =} \KeywordTok{c}\NormalTok{(}\StringTok{"Predicted"}\NormalTok{, }\StringTok{"Actual"}\NormalTok{), }\DataTypeTok{lty =} \DecValTok{1}\NormalTok{, }\DataTypeTok{col =} \KeywordTok{c}\NormalTok{(}\StringTok{"blue"}\NormalTok{,}\StringTok{"red"}\NormalTok{))}
\KeywordTok{lines}\NormalTok{(}\KeywordTok{fitted}\NormalTok{(arma2),}\DataTypeTok{col=}\StringTok{"blue"}\NormalTok{)}
\end{Highlighting}
\end{Shaded}

\includegraphics{Hw4_files/figure-latex/unnamed-chunk-4-2.pdf}

\begin{Shaded}
\begin{Highlighting}[]
\KeywordTok{plot}\NormalTok{(arma3}\OperatorTok{$}\NormalTok{x,}\DataTypeTok{col=}\StringTok{"red"}\NormalTok{, }\DataTypeTok{main =} \StringTok{"Plot of the third model and the time series"}\NormalTok{, }\DataTypeTok{xlab =} \StringTok{"Time"}\NormalTok{, }\DataTypeTok{ylab =} \StringTok{"Predicted and actual values"}\NormalTok{)}
\KeywordTok{legend}\NormalTok{(}\StringTok{"topright"}\NormalTok{, }\DataTypeTok{legend =} \KeywordTok{c}\NormalTok{(}\StringTok{"Predicted"}\NormalTok{, }\StringTok{"Actual"}\NormalTok{), }\DataTypeTok{lty =} \DecValTok{1}\NormalTok{, }\DataTypeTok{col =} \KeywordTok{c}\NormalTok{(}\StringTok{"blue"}\NormalTok{,}\StringTok{"red"}\NormalTok{))}
\KeywordTok{lines}\NormalTok{(}\KeywordTok{fitted}\NormalTok{(arma3),}\DataTypeTok{col=}\StringTok{"blue"}\NormalTok{)}
\end{Highlighting}
\end{Shaded}

\includegraphics{Hw4_files/figure-latex/unnamed-chunk-4-3.pdf}

All 3 models seem to fit the data pretty well. We would need to look at
the ACF and the PACF of all 3 models to see which one fits best.

\begin{enumerate}
\def\labelenumi{\alph{enumi})}
\setcounter{enumi}{2}
\tightlist
\item
  Running 3 models: (1) ARMA(2,0) and S-ARMA(0,3) with frequency 6, (2)
  ARMA(3,0) and S-ARMA(0,3) with frequency 5, (3)ARMA(3,0) and
  S-ARMA(0,2) with frequency 6
\end{enumerate}

\begin{Shaded}
\begin{Highlighting}[]
\CommentTok{#Look at the ACF and the PACF of residuals from the models and run a box test}
\KeywordTok{acf}\NormalTok{(arma1}\OperatorTok{$}\NormalTok{residuals, }\DataTypeTok{main =} \StringTok{"ACF of residual from first model"}\NormalTok{)}
\end{Highlighting}
\end{Shaded}

\includegraphics{Hw4_files/figure-latex/unnamed-chunk-5-1.pdf}

\begin{Shaded}
\begin{Highlighting}[]
\KeywordTok{pacf}\NormalTok{(arma1}\OperatorTok{$}\NormalTok{residuals, }\DataTypeTok{main =} \StringTok{"PACF of residual from first model"}\NormalTok{)}
\end{Highlighting}
\end{Shaded}

\includegraphics{Hw4_files/figure-latex/unnamed-chunk-5-2.pdf}

\begin{Shaded}
\begin{Highlighting}[]
\KeywordTok{acf}\NormalTok{(arma2}\OperatorTok{$}\NormalTok{residuals, }\DataTypeTok{main =} \StringTok{"ACF of residual from second model"}\NormalTok{)}
\end{Highlighting}
\end{Shaded}

\includegraphics{Hw4_files/figure-latex/unnamed-chunk-5-3.pdf}

\begin{Shaded}
\begin{Highlighting}[]
\KeywordTok{pacf}\NormalTok{(arma2}\OperatorTok{$}\NormalTok{residuals, }\DataTypeTok{main =} \StringTok{"PACF of residual from second model"}\NormalTok{)}
\end{Highlighting}
\end{Shaded}

\includegraphics{Hw4_files/figure-latex/unnamed-chunk-5-4.pdf}

\begin{Shaded}
\begin{Highlighting}[]
\KeywordTok{acf}\NormalTok{(arma2}\OperatorTok{$}\NormalTok{residuals, }\DataTypeTok{main =} \StringTok{"ACF of residual from third model"}\NormalTok{)}
\end{Highlighting}
\end{Shaded}

\includegraphics{Hw4_files/figure-latex/unnamed-chunk-5-5.pdf}

\begin{Shaded}
\begin{Highlighting}[]
\KeywordTok{pacf}\NormalTok{(arma2}\OperatorTok{$}\NormalTok{residuals, }\DataTypeTok{main =} \StringTok{"PACF of residual from third model"}\NormalTok{)}
\end{Highlighting}
\end{Shaded}

\includegraphics{Hw4_files/figure-latex/unnamed-chunk-5-6.pdf}

\begin{Shaded}
\begin{Highlighting}[]
\KeywordTok{Box.test}\NormalTok{(arma1}\OperatorTok{$}\NormalTok{residuals, }\DataTypeTok{lag =} \DecValTok{20}\NormalTok{)}
\end{Highlighting}
\end{Shaded}

\begin{verbatim}
## 
##  Box-Pierce test
## 
## data:  arma1$residuals
## X-squared = 53.484, df = 20, p-value = 6.897e-05
\end{verbatim}

\begin{Shaded}
\begin{Highlighting}[]
\KeywordTok{Box.test}\NormalTok{(arma2}\OperatorTok{$}\NormalTok{residuals, }\DataTypeTok{lag =} \DecValTok{20}\NormalTok{)}
\end{Highlighting}
\end{Shaded}

\begin{verbatim}
## 
##  Box-Pierce test
## 
## data:  arma2$residuals
## X-squared = 46.645, df = 20, p-value = 0.000657
\end{verbatim}

\begin{Shaded}
\begin{Highlighting}[]
\KeywordTok{Box.test}\NormalTok{(arma3}\OperatorTok{$}\NormalTok{residuals, }\DataTypeTok{lag =} \DecValTok{20}\NormalTok{)}
\end{Highlighting}
\end{Shaded}

\begin{verbatim}
## 
##  Box-Pierce test
## 
## data:  arma3$residuals
## X-squared = 49.224, df = 20, p-value = 0.0002857
\end{verbatim}

All three models seem to fit the data well, with the residuals
practically reduced to white noise looking at the ACF and the PACF as
well as the Box-Pierce test. However, since by the Box-Pierce test, the
first model has the lowest p-values, we'll take that model.

\begin{enumerate}
\def\labelenumi{\alph{enumi})}
\setcounter{enumi}{3}
\item
\end{enumerate}

\begin{Shaded}
\begin{Highlighting}[]
\KeywordTok{library}\NormalTok{(strucchange)}
\end{Highlighting}
\end{Shaded}

\begin{verbatim}
## Loading required package: zoo
\end{verbatim}

\begin{verbatim}
## 
## Attaching package: 'zoo'
\end{verbatim}

\begin{verbatim}
## The following object is masked from 'package:timeSeries':
## 
##     time<-
\end{verbatim}

\begin{verbatim}
## The following objects are masked from 'package:base':
## 
##     as.Date, as.Date.numeric
\end{verbatim}

\begin{verbatim}
## Loading required package: sandwich
\end{verbatim}

\begin{Shaded}
\begin{Highlighting}[]
\NormalTok{y=}\KeywordTok{recresid}\NormalTok{(arma1}\OperatorTok{$}\NormalTok{residuals}\OperatorTok{~}\DecValTok{1}\NormalTok{)}
\KeywordTok{plot}\NormalTok{(y, }\DataTypeTok{pch=}\DecValTok{16}\NormalTok{,}\DataTypeTok{ylab=}\StringTok{"Recursive Residuals"}\NormalTok{)}
\end{Highlighting}
\end{Shaded}

\includegraphics{Hw4_files/figure-latex/unnamed-chunk-6-1.pdf}

The recursive residuals suggests that there might a structural break of
my ARMA model predicting the data. This occurs at the middle of the
years, maybe around the year 1982. But we need to be sure by looking at
its cummulative sum.

\begin{enumerate}
\def\labelenumi{\alph{enumi})}
\setcounter{enumi}{4}
\item
\end{enumerate}

\begin{Shaded}
\begin{Highlighting}[]
\KeywordTok{plot}\NormalTok{(}\KeywordTok{efp}\NormalTok{(arma1}\OperatorTok{$}\NormalTok{residuals}\OperatorTok{~}\DecValTok{1}\NormalTok{, }\DataTypeTok{type =} \StringTok{"Rec-CUSUM"}\NormalTok{))}
\end{Highlighting}
\end{Shaded}

\includegraphics{Hw4_files/figure-latex/unnamed-chunk-7-1.pdf}

The recursive sum of the recursive residuals do not suggest that there
is a structural break, though there are irregularities in the middle.
Nevertheless, the model is robust enough to predict the data.

\begin{enumerate}
\def\labelenumi{\alph{enumi})}
\setcounter{enumi}{3}
\item
\end{enumerate}

\begin{Shaded}
\begin{Highlighting}[]
\CommentTok{#to obtain the best fit model according to R, we use auto.arima function}
\NormalTok{armaR =}\StringTok{ }\KeywordTok{auto.arima}\NormalTok{(ir_ts)}
\KeywordTok{summary}\NormalTok{(armaR)}
\end{Highlighting}
\end{Shaded}

\begin{verbatim}
## Series: ir_ts 
## ARIMA(1,1,2) 
## 
## Coefficients:
##          ar1      ma1      ma2
##       0.6284  -0.3065  -0.0527
## s.e.  0.0642   0.0675   0.0299
## 
## sigma^2 estimated as 0.03143:  log likelihood=768.32
## AIC=-1528.65   AICc=-1528.63   BIC=-1505.41
## 
## Training set error measures:
##                         ME      RMSE       MAE         MPE     MAPE
## Training set -0.0006210918 0.1771539 0.1046884 -0.05084758 1.820023
##                    MASE          ACF1
## Training set 0.07539674 -0.0002601686
\end{verbatim}

\begin{Shaded}
\begin{Highlighting}[]
\KeywordTok{summary}\NormalTok{(arma1)}
\end{Highlighting}
\end{Shaded}

\begin{verbatim}
## Series: ir_ts 
## ARIMA(2,0,0)(0,0,3)[6] with non-zero mean 
## 
## Coefficients:
##          ar1      ar2    sma1    sma2     sma3    mean
##       1.3448  -0.3472  0.0190  0.0201  -0.0485  6.0860
## s.e.  0.0189   0.0189  0.0204  0.0190   0.0194  1.3876
## 
## sigma^2 estimated as 0.03148:  log likelihood=765.3
## AIC=-1516.6   AICc=-1516.55   BIC=-1475.92
## 
## Training set error measures:
##                         ME      RMSE       MAE        MPE     MAPE
## Training set -0.0007431643 0.1772215 0.1049781 -0.1596363 1.833548
##                    MASE       ACF1
## Training set 0.07560533 -0.0202107
\end{verbatim}

The auto ARIMA function from R provides us the ARIMA(1,1,2) model to fit
the data. The AIC and the BIC seems to be similar for both the models.
So we can't say much from these 2 measures about which one is a better
model to fit the data. Thus, we want to look at the ACF and the PACF of
the residuals as well as conduct a Box-Pierce test to look at time
dependence.

\begin{Shaded}
\begin{Highlighting}[]
\KeywordTok{acf}\NormalTok{(armaR}\OperatorTok{$}\NormalTok{residuals, }\DataTypeTok{main =} \StringTok{"ACF of Auto ARIMA of interest rate"}\NormalTok{)}
\end{Highlighting}
\end{Shaded}

\includegraphics{Hw4_files/figure-latex/unnamed-chunk-9-1.pdf}

\begin{Shaded}
\begin{Highlighting}[]
\KeywordTok{pacf}\NormalTok{(armaR}\OperatorTok{$}\NormalTok{residuals, }\DataTypeTok{main =} \StringTok{"PACF of Auto ARIMA of interest rate"}\NormalTok{)}
\end{Highlighting}
\end{Shaded}

\includegraphics{Hw4_files/figure-latex/unnamed-chunk-9-2.pdf}

\begin{Shaded}
\begin{Highlighting}[]
\KeywordTok{Box.test}\NormalTok{(armaR}\OperatorTok{$}\NormalTok{residuals, }\DataTypeTok{lag =} \DecValTok{20}\NormalTok{)}
\end{Highlighting}
\end{Shaded}

\begin{verbatim}
## 
##  Box-Pierce test
## 
## data:  armaR$residuals
## X-squared = 49.591, df = 20, p-value = 0.0002534
\end{verbatim}

The ACF and PACF seems to have been reduced to that of a white noise,
which means that the ARIMA(1,1,2) took care of all the dynamics in the
data as well. The Box-Pierce seems to also conclude that result.
However, from the ACF and PACF of my model seems to suggest an even less
spikes in the plots, as well as an even lower P-value for the
Box-Pierce. This sugggests that my model seems to be doing better than
what `R' suggested but the models seem to be comparable in fitting the
data.

\begin{enumerate}
\def\labelenumi{\alph{enumi})}
\setcounter{enumi}{6}
\item
\end{enumerate}

\begin{Shaded}
\begin{Highlighting}[]
\CommentTok{#obtain a forecast object from each ARIMA model and then show its point forecast}
\NormalTok{arma1.forecast =}\StringTok{ }\KeywordTok{forecast}\NormalTok{(arma1, }\DataTypeTok{h=} \DecValTok{24}\NormalTok{)}
\NormalTok{arma1.forecast}\OperatorTok{$}\NormalTok{mean}
\end{Highlighting}
\end{Shaded}

\begin{verbatim}
## Time Series:
## Start = 2009.45600632244 
## End = 2009.89831401475 
## Frequency = 52 
##  [1] 0.6231658 0.6460586 0.6672275 0.6877722 0.7051344 0.7280438 0.7477210
##  [8] 0.7642885 0.7813322 0.7979600 0.8157029 0.8306220 0.8530309 0.8727072
## [15] 0.8959690 0.9188072 0.9393944 0.9589638 0.9780169 0.9968452 1.0155504
## [22] 1.0341681 1.0527110 1.0711835
\end{verbatim}

\begin{Shaded}
\begin{Highlighting}[]
\NormalTok{armaR.forecast =}\StringTok{ }\KeywordTok{forecast}\NormalTok{(armaR, }\DataTypeTok{h=} \DecValTok{24}\NormalTok{)}
\NormalTok{armaR.forecast}\OperatorTok{$}\NormalTok{mean}
\end{Highlighting}
\end{Shaded}

\begin{verbatim}
## Time Series:
## Start = 2009.45600632244 
## End = 2009.89831401475 
## Frequency = 52 
##  [1] 0.6038364 0.6048191 0.6054366 0.6058246 0.6060685 0.6062217 0.6063180
##  [8] 0.6063785 0.6064165 0.6064404 0.6064555 0.6064649 0.6064708 0.6064746
## [15] 0.6064769 0.6064784 0.6064793 0.6064799 0.6064802 0.6064805 0.6064806
## [22] 0.6064807 0.6064808 0.6064808
\end{verbatim}

We see that there seems to be a difference in the prediction. We take a
difference between these 2 forecasted values to look at how different
they are.

\begin{Shaded}
\begin{Highlighting}[]
\NormalTok{forecast.diff =}\StringTok{ }\NormalTok{arma1.forecast}\OperatorTok{$}\NormalTok{mean }\OperatorTok{-}\StringTok{ }\NormalTok{armaR.forecast}\OperatorTok{$}\NormalTok{mean}
\NormalTok{forecast.diff}
\end{Highlighting}
\end{Shaded}

\begin{verbatim}
## Time Series:
## Start = 2009.45600632244 
## End = 2009.89831401475 
## Frequency = 52 
##  [1] 0.01932942 0.04123949 0.06179092 0.08194758 0.09906596 0.12182211
##  [7] 0.14140294 0.15790998 0.17491566 0.19151959 0.20924748 0.22415715
## [13] 0.24656012 0.26623264 0.28949206 0.31232879 0.33291507 0.35248395
## [19] 0.37153667 0.39036471 0.40906976 0.42768741 0.44623022 0.46470271
\end{verbatim}

At every period, my model forecasted higher values. Not only that, there
also seems to be upward trend that my model is predicting, while the
model given by `R' seems to be fluctuating around 0.6.

\subsection{Problem 2}\label{problem-2}

Clear the previous environment and obtain the new data

\begin{Shaded}
\begin{Highlighting}[]
\KeywordTok{require}\NormalTok{(openxlsx)}
\end{Highlighting}
\end{Shaded}

\begin{verbatim}
## Loading required package: openxlsx
\end{verbatim}

\begin{Shaded}
\begin{Highlighting}[]
\KeywordTok{rm}\NormalTok{(}\DataTypeTok{list =} \KeywordTok{ls}\NormalTok{(}\DataTypeTok{all=}\NormalTok{T))}
\NormalTok{data =}\StringTok{ }\KeywordTok{read.xlsx}\NormalTok{(}\StringTok{"Chapter8_Exercises_Data.xlsx"}\NormalTok{, }\DataTypeTok{sheet =} \StringTok{"Exercise 7"}\NormalTok{)}
\KeywordTok{names}\NormalTok{(data)[}\DecValTok{4}\OperatorTok{:}\DecValTok{5}\NormalTok{] =}\StringTok{ }\KeywordTok{c}\NormalTok{(}\StringTok{"SP500_returns"}\NormalTok{, }\StringTok{"FTSE_returns"}\NormalTok{)}
\KeywordTok{attach}\NormalTok{(data)}
\end{Highlighting}
\end{Shaded}

In order to see if FTSE returns can influence SP500 returns, we want to
see if FTSE returns granger causes SP500 returns. But first we would
need to clean the data since there are many NA's in between. In order to
solve that problem, I would discard the data points for both returns if
only one of the datapoints is NA

\begin{Shaded}
\begin{Highlighting}[]
\NormalTok{index =}\StringTok{ }\KeywordTok{vector}\NormalTok{()}
\ControlFlowTok{for}\NormalTok{ (i }\ControlFlowTok{in} \DecValTok{1}\OperatorTok{:}\KeywordTok{length}\NormalTok{(SP500_returns))\{}
  \ControlFlowTok{if}\NormalTok{ (}\KeywordTok{is.na}\NormalTok{(SP500_returns[i]) }\OperatorTok{||}\StringTok{ }\KeywordTok{is.na}\NormalTok{(FTSE_returns[i]))}
\NormalTok{    index =}\StringTok{ }\KeywordTok{append}\NormalTok{(index, i)}
\NormalTok{\}}

\NormalTok{data =}\StringTok{ }\NormalTok{data[}\OperatorTok{-}\NormalTok{index,]}
\KeywordTok{attach}\NormalTok{(data)}
\end{Highlighting}
\end{Shaded}

\begin{verbatim}
## The following objects are masked from data (pos = 3):
## 
##     FTSE_OPEN, FTSE_returns, OBS, SP500_OPEN, SP500_returns
\end{verbatim}

After removing all the empty data points, we first find the order for
the VAR model.

\begin{Shaded}
\begin{Highlighting}[]
\KeywordTok{require}\NormalTok{(}\StringTok{"vars"}\NormalTok{)}
\end{Highlighting}
\end{Shaded}

\begin{verbatim}
## Loading required package: vars
\end{verbatim}

\begin{verbatim}
## Loading required package: MASS
\end{verbatim}

\begin{verbatim}
## Loading required package: urca
\end{verbatim}

\begin{verbatim}
## Loading required package: lmtest
\end{verbatim}

\begin{Shaded}
\begin{Highlighting}[]
\KeywordTok{require}\NormalTok{(}\StringTok{"VAR.etp"}\NormalTok{)}
\end{Highlighting}
\end{Shaded}

\begin{verbatim}
## Loading required package: VAR.etp
\end{verbatim}

\begin{Shaded}
\begin{Highlighting}[]
\NormalTok{SP500_ts =}\StringTok{ }\KeywordTok{ts}\NormalTok{(SP500_returns, }\DataTypeTok{start =} \DecValTok{1990}\NormalTok{, }\DataTypeTok{freq =} \DecValTok{252}\NormalTok{)}
\NormalTok{FTSE_ts =}\StringTok{ }\KeywordTok{ts}\NormalTok{(FTSE_returns, }\DataTypeTok{start =} \DecValTok{1990}\NormalTok{, }\DataTypeTok{freq =} \DecValTok{252}\NormalTok{)}
\NormalTok{y =}\StringTok{ }\KeywordTok{cbind}\NormalTok{(SP500_ts, FTSE_ts)}
\NormalTok{y_tot=}\KeywordTok{data.frame}\NormalTok{(y)}
\NormalTok{y_cri=}\KeywordTok{VAR.select}\NormalTok{(y_tot, }\DataTypeTok{pmax =} \DecValTok{10}\NormalTok{)}
\KeywordTok{print}\NormalTok{(y_cri}\OperatorTok{$}\NormalTok{p)}
\end{Highlighting}
\end{Shaded}

\begin{verbatim}
## [1] 8
\end{verbatim}

From VAR.select function, we find that the optimal order is 8. So we
select the order to be 8 and check Granger causality. But just in case,
we check for Granger causality of FTSE returns on SP500 returns

\begin{Shaded}
\begin{Highlighting}[]
\ControlFlowTok{for}\NormalTok{ (i }\ControlFlowTok{in} \DecValTok{1}\OperatorTok{:}\DecValTok{8}\NormalTok{)\{}
\KeywordTok{print}\NormalTok{(}\KeywordTok{grangertest}\NormalTok{(FTSE_returns }\OperatorTok{~}\StringTok{ }\NormalTok{SP500_returns, }\DataTypeTok{order =}\NormalTok{ i))}
\KeywordTok{print}\NormalTok{(}\KeywordTok{grangertest}\NormalTok{(SP500_returns}\OperatorTok{~}\NormalTok{FTSE_returns , }\DataTypeTok{order =}\NormalTok{ i))}
\NormalTok{\}}
\end{Highlighting}
\end{Shaded}

\begin{verbatim}
## Granger causality test
## 
## Model 1: FTSE_returns ~ Lags(FTSE_returns, 1:1) + Lags(SP500_returns, 1:1)
## Model 2: FTSE_returns ~ Lags(FTSE_returns, 1:1)
##   Res.Df Df      F    Pr(>F)    
## 1   5348                        
## 2   5349 -1 516.64 < 2.2e-16 ***
## ---
## Signif. codes:  0 '***' 0.001 '**' 0.01 '*' 0.05 '.' 0.1 ' ' 1
## Granger causality test
## 
## Model 1: SP500_returns ~ Lags(SP500_returns, 1:1) + Lags(FTSE_returns, 1:1)
## Model 2: SP500_returns ~ Lags(SP500_returns, 1:1)
##   Res.Df Df      F Pr(>F)
## 1   5348                 
## 2   5349 -1 1.0704 0.3009
## Granger causality test
## 
## Model 1: FTSE_returns ~ Lags(FTSE_returns, 1:2) + Lags(SP500_returns, 1:2)
## Model 2: FTSE_returns ~ Lags(FTSE_returns, 1:2)
##   Res.Df Df      F    Pr(>F)    
## 1   5345                        
## 2   5347 -2 265.68 < 2.2e-16 ***
## ---
## Signif. codes:  0 '***' 0.001 '**' 0.01 '*' 0.05 '.' 0.1 ' ' 1
## Granger causality test
## 
## Model 1: SP500_returns ~ Lags(SP500_returns, 1:2) + Lags(FTSE_returns, 1:2)
## Model 2: SP500_returns ~ Lags(SP500_returns, 1:2)
##   Res.Df Df     F  Pr(>F)  
## 1   5345                   
## 2   5347 -2 2.539 0.07904 .
## ---
## Signif. codes:  0 '***' 0.001 '**' 0.01 '*' 0.05 '.' 0.1 ' ' 1
## Granger causality test
## 
## Model 1: FTSE_returns ~ Lags(FTSE_returns, 1:3) + Lags(SP500_returns, 1:3)
## Model 2: FTSE_returns ~ Lags(FTSE_returns, 1:3)
##   Res.Df Df      F    Pr(>F)    
## 1   5342                        
## 2   5345 -3 179.81 < 2.2e-16 ***
## ---
## Signif. codes:  0 '***' 0.001 '**' 0.01 '*' 0.05 '.' 0.1 ' ' 1
## Granger causality test
## 
## Model 1: SP500_returns ~ Lags(SP500_returns, 1:3) + Lags(FTSE_returns, 1:3)
## Model 2: SP500_returns ~ Lags(SP500_returns, 1:3)
##   Res.Df Df      F Pr(>F)
## 1   5342                 
## 2   5345 -3 1.6741 0.1703
## Granger causality test
## 
## Model 1: FTSE_returns ~ Lags(FTSE_returns, 1:4) + Lags(SP500_returns, 1:4)
## Model 2: FTSE_returns ~ Lags(FTSE_returns, 1:4)
##   Res.Df Df      F    Pr(>F)    
## 1   5339                        
## 2   5343 -4 133.74 < 2.2e-16 ***
## ---
## Signif. codes:  0 '***' 0.001 '**' 0.01 '*' 0.05 '.' 0.1 ' ' 1
## Granger causality test
## 
## Model 1: SP500_returns ~ Lags(SP500_returns, 1:4) + Lags(FTSE_returns, 1:4)
## Model 2: SP500_returns ~ Lags(SP500_returns, 1:4)
##   Res.Df Df      F Pr(>F)
## 1   5339                 
## 2   5343 -4 1.4657 0.2098
## Granger causality test
## 
## Model 1: FTSE_returns ~ Lags(FTSE_returns, 1:5) + Lags(SP500_returns, 1:5)
## Model 2: FTSE_returns ~ Lags(FTSE_returns, 1:5)
##   Res.Df Df      F    Pr(>F)    
## 1   5336                        
## 2   5341 -5 106.71 < 2.2e-16 ***
## ---
## Signif. codes:  0 '***' 0.001 '**' 0.01 '*' 0.05 '.' 0.1 ' ' 1
## Granger causality test
## 
## Model 1: SP500_returns ~ Lags(SP500_returns, 1:5) + Lags(FTSE_returns, 1:5)
## Model 2: SP500_returns ~ Lags(SP500_returns, 1:5)
##   Res.Df Df      F   Pr(>F)   
## 1   5336                      
## 2   5341 -5 3.3913 0.004626 **
## ---
## Signif. codes:  0 '***' 0.001 '**' 0.01 '*' 0.05 '.' 0.1 ' ' 1
## Granger causality test
## 
## Model 1: FTSE_returns ~ Lags(FTSE_returns, 1:6) + Lags(SP500_returns, 1:6)
## Model 2: FTSE_returns ~ Lags(FTSE_returns, 1:6)
##   Res.Df Df      F    Pr(>F)    
## 1   5333                        
## 2   5339 -6 88.104 < 2.2e-16 ***
## ---
## Signif. codes:  0 '***' 0.001 '**' 0.01 '*' 0.05 '.' 0.1 ' ' 1
## Granger causality test
## 
## Model 1: SP500_returns ~ Lags(SP500_returns, 1:6) + Lags(FTSE_returns, 1:6)
## Model 2: SP500_returns ~ Lags(SP500_returns, 1:6)
##   Res.Df Df      F   Pr(>F)   
## 1   5333                      
## 2   5339 -6 3.0756 0.005245 **
## ---
## Signif. codes:  0 '***' 0.001 '**' 0.01 '*' 0.05 '.' 0.1 ' ' 1
## Granger causality test
## 
## Model 1: FTSE_returns ~ Lags(FTSE_returns, 1:7) + Lags(SP500_returns, 1:7)
## Model 2: FTSE_returns ~ Lags(FTSE_returns, 1:7)
##   Res.Df Df      F    Pr(>F)    
## 1   5330                        
## 2   5337 -7 75.635 < 2.2e-16 ***
## ---
## Signif. codes:  0 '***' 0.001 '**' 0.01 '*' 0.05 '.' 0.1 ' ' 1
## Granger causality test
## 
## Model 1: SP500_returns ~ Lags(SP500_returns, 1:7) + Lags(FTSE_returns, 1:7)
## Model 2: SP500_returns ~ Lags(SP500_returns, 1:7)
##   Res.Df Df      F    Pr(>F)    
## 1   5330                        
## 2   5337 -7 4.5631 4.309e-05 ***
## ---
## Signif. codes:  0 '***' 0.001 '**' 0.01 '*' 0.05 '.' 0.1 ' ' 1
## Granger causality test
## 
## Model 1: FTSE_returns ~ Lags(FTSE_returns, 1:8) + Lags(SP500_returns, 1:8)
## Model 2: FTSE_returns ~ Lags(FTSE_returns, 1:8)
##   Res.Df Df      F    Pr(>F)    
## 1   5327                        
## 2   5335 -8 65.965 < 2.2e-16 ***
## ---
## Signif. codes:  0 '***' 0.001 '**' 0.01 '*' 0.05 '.' 0.1 ' ' 1
## Granger causality test
## 
## Model 1: SP500_returns ~ Lags(SP500_returns, 1:8) + Lags(FTSE_returns, 1:8)
## Model 2: SP500_returns ~ Lags(SP500_returns, 1:8)
##   Res.Df Df      F    Pr(>F)    
## 1   5327                        
## 2   5335 -8 3.5615 0.0003987 ***
## ---
## Signif. codes:  0 '***' 0.001 '**' 0.01 '*' 0.05 '.' 0.1 ' ' 1
\end{verbatim}

From the Granger causality test, we could see for sure that SP500
returns does influence FTSE returns, while FTSE returns affect SP500
returns as the number of lags is more than 4. Maybe because as time goes
by, the eventual shock on the FTSE market would eventually linger to the
SP500. Thus, we can use SP500 returns to forecase FTSE returns, and also
the other way around for lags more than 4.

\subsection{Problem 3}\label{problem-3}

Clear the previous environment and add in the new data

\begin{Shaded}
\begin{Highlighting}[]
\KeywordTok{rm}\NormalTok{(}\DataTypeTok{list =} \KeywordTok{ls}\NormalTok{(}\DataTypeTok{all=}\NormalTok{T))}
\NormalTok{data =}\StringTok{ }\KeywordTok{read.xlsx}\NormalTok{(}\StringTok{"Chapter11_exercises_data.xlsx"}\NormalTok{)}
\NormalTok{data =}\StringTok{ }\NormalTok{data[,}\OperatorTok{-}\DecValTok{8}\NormalTok{]}
\KeywordTok{attach}\NormalTok{(data)}
\end{Highlighting}
\end{Shaded}

Construct 4 VAR models, which are of GSF and GSJ(VAR\_1), GSF and
GAL(VAR\_2), GSj and GAL(VAR\_3), and GSF, GSJ and GAL
together(VAR\_4).(All these 3 symbols are quarterly house price growth
rate based on original Freddie Mac House Price Indices of San
Francisco-Oakland-Freemont, San Jose-Sunnyvale-Santa Clara,
Albany-Schenectady-Oakland-Freemont respectively)

\begin{Shaded}
\begin{Highlighting}[]
\NormalTok{GSF_ts =}\StringTok{ }\KeywordTok{ts}\NormalTok{(GSF[}\KeywordTok{complete.cases}\NormalTok{(GSF)], }\DataTypeTok{start =} \DecValTok{1957}\NormalTok{, }\DataTypeTok{freq =} \DecValTok{4}\NormalTok{)}
\NormalTok{GSJ_ts =}\StringTok{ }\KeywordTok{ts}\NormalTok{(GSJ[}\KeywordTok{complete.cases}\NormalTok{(GSJ)], }\DataTypeTok{start =} \DecValTok{1957}\NormalTok{, }\DataTypeTok{freq =} \DecValTok{4}\NormalTok{)}
\NormalTok{GAL_ts =}\StringTok{ }\KeywordTok{ts}\NormalTok{(GAL[}\KeywordTok{complete.cases}\NormalTok{(GAL)], }\DataTypeTok{start =} \DecValTok{1957}\NormalTok{, }\DataTypeTok{freq =} \DecValTok{4}\NormalTok{)}

\NormalTok{y1 =}\StringTok{ }\KeywordTok{cbind}\NormalTok{(GSF_ts, GSJ_ts)}
\NormalTok{y2 =}\StringTok{ }\KeywordTok{cbind}\NormalTok{(GSF_ts, GAL_ts)}
\NormalTok{y3 =}\StringTok{ }\KeywordTok{cbind}\NormalTok{(GSJ_ts, GAL_ts)}
\NormalTok{y4 =}\StringTok{ }\KeywordTok{cbind}\NormalTok{(GSF_ts, GSJ_ts, GAL_ts)}

\NormalTok{y_tot1=}\KeywordTok{data.frame}\NormalTok{(y1)}
\NormalTok{y_cri1=}\KeywordTok{VAR.select}\NormalTok{(y_tot1, }\DataTypeTok{pmax =} \DecValTok{20}\NormalTok{)}
\NormalTok{y_tot2=}\KeywordTok{data.frame}\NormalTok{(y2)}
\NormalTok{y_cri2=}\KeywordTok{VAR.select}\NormalTok{(y_tot2, }\DataTypeTok{pmax =} \DecValTok{10}\NormalTok{)}
\NormalTok{y_tot3=}\KeywordTok{data.frame}\NormalTok{(y3)}
\NormalTok{y_cri3=}\KeywordTok{VAR.select}\NormalTok{(y_tot3, }\DataTypeTok{pmax =} \DecValTok{10}\NormalTok{)}
\NormalTok{y_tot4=}\KeywordTok{data.frame}\NormalTok{(y4)}
\NormalTok{y_cri4=}\KeywordTok{VAR.select}\NormalTok{(y_tot4, }\DataTypeTok{pmax =} \DecValTok{10}\NormalTok{)}

\KeywordTok{print}\NormalTok{(y_cri1}\OperatorTok{$}\NormalTok{p)}
\end{Highlighting}
\end{Shaded}

\begin{verbatim}
## [1] 3
\end{verbatim}

\begin{Shaded}
\begin{Highlighting}[]
\KeywordTok{print}\NormalTok{(y_cri2}\OperatorTok{$}\NormalTok{p)}
\end{Highlighting}
\end{Shaded}

\begin{verbatim}
## [1] 7
\end{verbatim}

\begin{Shaded}
\begin{Highlighting}[]
\KeywordTok{print}\NormalTok{(y_cri3}\OperatorTok{$}\NormalTok{p)}
\end{Highlighting}
\end{Shaded}

\begin{verbatim}
## [1] 7
\end{verbatim}

\begin{Shaded}
\begin{Highlighting}[]
\KeywordTok{print}\NormalTok{(y_cri4}\OperatorTok{$}\NormalTok{p)}
\end{Highlighting}
\end{Shaded}

\begin{verbatim}
## [1] 3
\end{verbatim}

This shows that the order to use for VAR\_1, VAR\_2, VAR\_3 and VAR\_4
are 3, 7, 7 and 3 respectively.

\begin{Shaded}
\begin{Highlighting}[]
\NormalTok{VAR_}\DecValTok{1}\NormalTok{ =}\StringTok{ }\KeywordTok{VAR}\NormalTok{(y_tot1, }\DataTypeTok{p =} \DecValTok{3}\NormalTok{)}
\NormalTok{VAR_}\DecValTok{2}\NormalTok{ =}\StringTok{ }\KeywordTok{VAR}\NormalTok{(y_tot2, }\DataTypeTok{p =} \DecValTok{7}\NormalTok{)}
\NormalTok{VAR_}\DecValTok{3}\NormalTok{ =}\StringTok{ }\KeywordTok{VAR}\NormalTok{(y_tot3, }\DataTypeTok{p =} \DecValTok{7}\NormalTok{)}
\NormalTok{VAR_}\DecValTok{4}\NormalTok{ =}\StringTok{ }\KeywordTok{VAR}\NormalTok{(y_tot4, }\DataTypeTok{p =} \DecValTok{3}\NormalTok{)}

\NormalTok{VAR_}\FloatTok{1.}\NormalTok{predict =}\StringTok{ }\KeywordTok{predict}\NormalTok{(}\DataTypeTok{object=}\NormalTok{VAR_}\DecValTok{1}\NormalTok{, }\DataTypeTok{n.ahead=}\DecValTok{1}\NormalTok{)}
\NormalTok{VAR_}\FloatTok{2.}\NormalTok{predict =}\StringTok{ }\KeywordTok{predict}\NormalTok{(}\DataTypeTok{object=}\NormalTok{VAR_}\DecValTok{2}\NormalTok{, }\DataTypeTok{n.ahead=}\DecValTok{1}\NormalTok{)}
\NormalTok{VAR_}\FloatTok{3.}\NormalTok{predict =}\StringTok{ }\KeywordTok{predict}\NormalTok{(}\DataTypeTok{object=}\NormalTok{VAR_}\DecValTok{3}\NormalTok{, }\DataTypeTok{n.ahead=}\DecValTok{1}\NormalTok{)}
\NormalTok{VAR_}\FloatTok{4.}\NormalTok{predict =}\StringTok{ }\KeywordTok{predict}\NormalTok{(}\DataTypeTok{object=}\NormalTok{VAR_}\DecValTok{4}\NormalTok{, }\DataTypeTok{n.ahead=}\DecValTok{1}\NormalTok{)}

\KeywordTok{print}\NormalTok{(VAR_}\FloatTok{1.}\NormalTok{predict}\OperatorTok{$}\NormalTok{fcst)}
\end{Highlighting}
\end{Shaded}

\begin{verbatim}
## $GSF_ts
##                  fcst     lower    upper       CI
## GSF_ts.fcst -2.102955 -6.075083 1.869172 3.972127
## 
## $GSJ_ts
##                  fcst     lower    upper      CI
## GSJ_ts.fcst -2.105309 -5.317858 1.107241 3.21255
\end{verbatim}

\begin{Shaded}
\begin{Highlighting}[]
\KeywordTok{print}\NormalTok{(VAR_}\FloatTok{2.}\NormalTok{predict}\OperatorTok{$}\NormalTok{fcst)}
\end{Highlighting}
\end{Shaded}

\begin{verbatim}
## $GSF_ts
##                  fcst    lower    upper       CI
## GSF_ts.fcst -0.989752 -4.98998 3.010476 4.000228
## 
## $GAL_ts
##                   fcst     lower    upper       CI
## GAL_ts.fcst -0.7083885 -4.544204 3.127426 3.835815
\end{verbatim}

\begin{Shaded}
\begin{Highlighting}[]
\KeywordTok{print}\NormalTok{(VAR_}\FloatTok{3.}\NormalTok{predict}\OperatorTok{$}\NormalTok{fcst)}
\end{Highlighting}
\end{Shaded}

\begin{verbatim}
## $GSJ_ts
##                   fcst     lower    upper       CI
## GSJ_ts.fcst -0.3327676 -3.481943 2.816408 3.149175
## 
## $GAL_ts
##                   fcst     lower    upper      CI
## GAL_ts.fcst -0.2545135 -4.175454 3.666427 3.92094
\end{verbatim}

\begin{Shaded}
\begin{Highlighting}[]
\KeywordTok{print}\NormalTok{(VAR_}\FloatTok{4.}\NormalTok{predict}\OperatorTok{$}\NormalTok{fcst)}
\end{Highlighting}
\end{Shaded}

\begin{verbatim}
## $GSF_ts
##                  fcst     lower    upper       CI
## GSF_ts.fcst -2.126521 -6.058267 1.805225 3.931746
## 
## $GSJ_ts
##                  fcst     lower    upper       CI
## GSJ_ts.fcst -2.066979 -5.244068 1.110111 3.177089
## 
## $GAL_ts
##                  fcst     lower    upper       CI
## GAL_ts.fcst -1.074373 -5.133941 2.985194 4.059568
\end{verbatim}

We see that with or without adding the GAL, the forecasted values of GSF
and GSJ are similar. That implies that since
Albany-Schenectady-Oakland-Freemont is so far from the other 2, there is
no significant effect on the forecast.


\end{document}
