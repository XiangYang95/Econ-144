\documentclass[]{article}
\usepackage{lmodern}
\usepackage{amssymb,amsmath}
\usepackage{ifxetex,ifluatex}
\usepackage{fixltx2e} % provides \textsubscript
\ifnum 0\ifxetex 1\fi\ifluatex 1\fi=0 % if pdftex
  \usepackage[T1]{fontenc}
  \usepackage[utf8]{inputenc}
\else % if luatex or xelatex
  \ifxetex
    \usepackage{mathspec}
  \else
    \usepackage{fontspec}
  \fi
  \defaultfontfeatures{Ligatures=TeX,Scale=MatchLowercase}
\fi
% use upquote if available, for straight quotes in verbatim environments
\IfFileExists{upquote.sty}{\usepackage{upquote}}{}
% use microtype if available
\IfFileExists{microtype.sty}{%
\usepackage{microtype}
\UseMicrotypeSet[protrusion]{basicmath} % disable protrusion for tt fonts
}{}
\usepackage[margin=1in]{geometry}
\usepackage{hyperref}
\hypersetup{unicode=true,
            pdftitle={Homework1.R},
            pdfauthor={Lenovo},
            pdfborder={0 0 0},
            breaklinks=true}
\urlstyle{same}  % don't use monospace font for urls
\usepackage{color}
\usepackage{fancyvrb}
\newcommand{\VerbBar}{|}
\newcommand{\VERB}{\Verb[commandchars=\\\{\}]}
\DefineVerbatimEnvironment{Highlighting}{Verbatim}{commandchars=\\\{\}}
% Add ',fontsize=\small' for more characters per line
\usepackage{framed}
\definecolor{shadecolor}{RGB}{248,248,248}
\newenvironment{Shaded}{\begin{snugshade}}{\end{snugshade}}
\newcommand{\KeywordTok}[1]{\textcolor[rgb]{0.13,0.29,0.53}{\textbf{#1}}}
\newcommand{\DataTypeTok}[1]{\textcolor[rgb]{0.13,0.29,0.53}{#1}}
\newcommand{\DecValTok}[1]{\textcolor[rgb]{0.00,0.00,0.81}{#1}}
\newcommand{\BaseNTok}[1]{\textcolor[rgb]{0.00,0.00,0.81}{#1}}
\newcommand{\FloatTok}[1]{\textcolor[rgb]{0.00,0.00,0.81}{#1}}
\newcommand{\ConstantTok}[1]{\textcolor[rgb]{0.00,0.00,0.00}{#1}}
\newcommand{\CharTok}[1]{\textcolor[rgb]{0.31,0.60,0.02}{#1}}
\newcommand{\SpecialCharTok}[1]{\textcolor[rgb]{0.00,0.00,0.00}{#1}}
\newcommand{\StringTok}[1]{\textcolor[rgb]{0.31,0.60,0.02}{#1}}
\newcommand{\VerbatimStringTok}[1]{\textcolor[rgb]{0.31,0.60,0.02}{#1}}
\newcommand{\SpecialStringTok}[1]{\textcolor[rgb]{0.31,0.60,0.02}{#1}}
\newcommand{\ImportTok}[1]{#1}
\newcommand{\CommentTok}[1]{\textcolor[rgb]{0.56,0.35,0.01}{\textit{#1}}}
\newcommand{\DocumentationTok}[1]{\textcolor[rgb]{0.56,0.35,0.01}{\textbf{\textit{#1}}}}
\newcommand{\AnnotationTok}[1]{\textcolor[rgb]{0.56,0.35,0.01}{\textbf{\textit{#1}}}}
\newcommand{\CommentVarTok}[1]{\textcolor[rgb]{0.56,0.35,0.01}{\textbf{\textit{#1}}}}
\newcommand{\OtherTok}[1]{\textcolor[rgb]{0.56,0.35,0.01}{#1}}
\newcommand{\FunctionTok}[1]{\textcolor[rgb]{0.00,0.00,0.00}{#1}}
\newcommand{\VariableTok}[1]{\textcolor[rgb]{0.00,0.00,0.00}{#1}}
\newcommand{\ControlFlowTok}[1]{\textcolor[rgb]{0.13,0.29,0.53}{\textbf{#1}}}
\newcommand{\OperatorTok}[1]{\textcolor[rgb]{0.81,0.36,0.00}{\textbf{#1}}}
\newcommand{\BuiltInTok}[1]{#1}
\newcommand{\ExtensionTok}[1]{#1}
\newcommand{\PreprocessorTok}[1]{\textcolor[rgb]{0.56,0.35,0.01}{\textit{#1}}}
\newcommand{\AttributeTok}[1]{\textcolor[rgb]{0.77,0.63,0.00}{#1}}
\newcommand{\RegionMarkerTok}[1]{#1}
\newcommand{\InformationTok}[1]{\textcolor[rgb]{0.56,0.35,0.01}{\textbf{\textit{#1}}}}
\newcommand{\WarningTok}[1]{\textcolor[rgb]{0.56,0.35,0.01}{\textbf{\textit{#1}}}}
\newcommand{\AlertTok}[1]{\textcolor[rgb]{0.94,0.16,0.16}{#1}}
\newcommand{\ErrorTok}[1]{\textcolor[rgb]{0.64,0.00,0.00}{\textbf{#1}}}
\newcommand{\NormalTok}[1]{#1}
\usepackage{graphicx,grffile}
\makeatletter
\def\maxwidth{\ifdim\Gin@nat@width>\linewidth\linewidth\else\Gin@nat@width\fi}
\def\maxheight{\ifdim\Gin@nat@height>\textheight\textheight\else\Gin@nat@height\fi}
\makeatother
% Scale images if necessary, so that they will not overflow the page
% margins by default, and it is still possible to overwrite the defaults
% using explicit options in \includegraphics[width, height, ...]{}
\setkeys{Gin}{width=\maxwidth,height=\maxheight,keepaspectratio}
\IfFileExists{parskip.sty}{%
\usepackage{parskip}
}{% else
\setlength{\parindent}{0pt}
\setlength{\parskip}{6pt plus 2pt minus 1pt}
}
\setlength{\emergencystretch}{3em}  % prevent overfull lines
\providecommand{\tightlist}{%
  \setlength{\itemsep}{0pt}\setlength{\parskip}{0pt}}
\setcounter{secnumdepth}{0}
% Redefines (sub)paragraphs to behave more like sections
\ifx\paragraph\undefined\else
\let\oldparagraph\paragraph
\renewcommand{\paragraph}[1]{\oldparagraph{#1}\mbox{}}
\fi
\ifx\subparagraph\undefined\else
\let\oldsubparagraph\subparagraph
\renewcommand{\subparagraph}[1]{\oldsubparagraph{#1}\mbox{}}
\fi

%%% Use protect on footnotes to avoid problems with footnotes in titles
\let\rmarkdownfootnote\footnote%
\def\footnote{\protect\rmarkdownfootnote}

%%% Change title format to be more compact
\usepackage{titling}

% Create subtitle command for use in maketitle
\newcommand{\subtitle}[1]{
  \posttitle{
    \begin{center}\large#1\end{center}
    }
}

\setlength{\droptitle}{-2em}
  \title{Homework1.R}
  \pretitle{\vspace{\droptitle}\centering\huge}
  \posttitle{\par}
  \author{Lenovo}
  \preauthor{\centering\large\emph}
  \postauthor{\par}
  \predate{\centering\large\emph}
  \postdate{\par}
  \date{Wed Apr 11 23:00:48 2018}


\begin{document}
\maketitle

\begin{Shaded}
\begin{Highlighting}[]
\CommentTok{#changed the working directory}
\KeywordTok{setwd}\NormalTok{(}\StringTok{"/Users/Lenovo/Desktop/Econ 144/Homework/Hw 1"}\NormalTok{)}
\CommentTok{#source("/Users/Lenovo/Desktop/Econ 144/Homework/Hw 1/Homework1.R")}

\CommentTok{#clear all variables and prior sessions}
\KeywordTok{rm}\NormalTok{(}\DataTypeTok{list=}\KeywordTok{ls}\NormalTok{(}\DataTypeTok{all=}\OtherTok{TRUE}\NormalTok{))}

\KeywordTok{library}\NormalTok{(DAAG)}
\end{Highlighting}
\end{Shaded}

\begin{verbatim}
## Loading required package: lattice
\end{verbatim}

\begin{Shaded}
\begin{Highlighting}[]
\KeywordTok{library}\NormalTok{(lattice)}
\KeywordTok{library}\NormalTok{(foreign)}
\KeywordTok{library}\NormalTok{(MASS)}
\end{Highlighting}
\end{Shaded}

\begin{verbatim}
## 
## Attaching package: 'MASS'
\end{verbatim}

\begin{verbatim}
## The following object is masked from 'package:DAAG':
## 
##     hills
\end{verbatim}

\begin{Shaded}
\begin{Highlighting}[]
\KeywordTok{library}\NormalTok{(car)}
\end{Highlighting}
\end{Shaded}

\begin{verbatim}
## Loading required package: carData
\end{verbatim}

\begin{verbatim}
## 
## Attaching package: 'car'
\end{verbatim}

\begin{verbatim}
## The following object is masked from 'package:DAAG':
## 
##     vif
\end{verbatim}

\begin{Shaded}
\begin{Highlighting}[]
\KeywordTok{require}\NormalTok{(stats)}
\KeywordTok{require}\NormalTok{(stats4)}
\end{Highlighting}
\end{Shaded}

\begin{verbatim}
## Loading required package: stats4
\end{verbatim}

\begin{Shaded}
\begin{Highlighting}[]
\KeywordTok{library}\NormalTok{(KernSmooth)}
\end{Highlighting}
\end{Shaded}

\begin{verbatim}
## KernSmooth 2.23 loaded
## Copyright M. P. Wand 1997-2009
\end{verbatim}

\begin{Shaded}
\begin{Highlighting}[]
\KeywordTok{library}\NormalTok{(fastICA)}
\KeywordTok{library}\NormalTok{(cluster)}
\KeywordTok{library}\NormalTok{(leaps)}
\KeywordTok{library}\NormalTok{(mgcv)}
\end{Highlighting}
\end{Shaded}

\begin{verbatim}
## Loading required package: nlme
\end{verbatim}

\begin{verbatim}
## This is mgcv 1.8-23. For overview type 'help("mgcv-package")'.
\end{verbatim}

\begin{Shaded}
\begin{Highlighting}[]
\KeywordTok{library}\NormalTok{(rpart)}
\KeywordTok{library}\NormalTok{(pan)}
\KeywordTok{library}\NormalTok{(mgcv)}
\KeywordTok{library}\NormalTok{(e1071)}
\KeywordTok{library}\NormalTok{(corrplot)}
\end{Highlighting}
\end{Shaded}

\begin{verbatim}
## corrplot 0.84 loaded
\end{verbatim}

\begin{Shaded}
\begin{Highlighting}[]
\CommentTok{#imported the dataset from DAAG'}
\NormalTok{data =}\StringTok{ }\KeywordTok{data.frame}\NormalTok{(nsw74psid1)}
\KeywordTok{attach}\NormalTok{(data)}
\KeywordTok{summary}\NormalTok{(nsw74psid1)}
\end{Highlighting}
\end{Shaded}

\begin{verbatim}
##       trt               age             educ           black       
##  Min.   :0.00000   Min.   :17.00   Min.   : 0.00   Min.   :0.0000  
##  1st Qu.:0.00000   1st Qu.:25.00   1st Qu.:10.00   1st Qu.:0.0000  
##  Median :0.00000   Median :32.00   Median :12.00   Median :0.0000  
##  Mean   :0.06916   Mean   :34.23   Mean   :11.99   Mean   :0.2916  
##  3rd Qu.:0.00000   3rd Qu.:43.50   3rd Qu.:14.00   3rd Qu.:1.0000  
##  Max.   :1.00000   Max.   :55.00   Max.   :17.00   Max.   :1.0000  
##       hisp              marr            nodeg             re74       
##  Min.   :0.00000   Min.   :0.0000   Min.   :0.0000   Min.   :     0  
##  1st Qu.:0.00000   1st Qu.:1.0000   1st Qu.:0.0000   1st Qu.:  8817  
##  Median :0.00000   Median :1.0000   Median :0.0000   Median : 17437  
##  Mean   :0.03439   Mean   :0.8194   Mean   :0.3331   Mean   : 18230  
##  3rd Qu.:0.00000   3rd Qu.:1.0000   3rd Qu.:1.0000   3rd Qu.: 25470  
##  Max.   :1.00000   Max.   :1.0000   Max.   :1.0000   Max.   :137149  
##       re75             re78       
##  Min.   :     0   Min.   :     0  
##  1st Qu.:  7605   1st Qu.:  9243  
##  Median : 17008   Median : 19432  
##  Mean   : 17851   Mean   : 20502  
##  3rd Qu.: 25584   3rd Qu.: 28816  
##  Max.   :156653   Max.   :121174
\end{verbatim}

\begin{Shaded}
\begin{Highlighting}[]
\NormalTok{##(a) Plot a histogram of each variable and discuss its properties.}
\KeywordTok{windows}\NormalTok{()}
\KeywordTok{par}\NormalTok{(}\DataTypeTok{mfrow=}\KeywordTok{c}\NormalTok{(}\DecValTok{2}\NormalTok{,}\DecValTok{2}\NormalTok{))}
\KeywordTok{truehist}\NormalTok{(trt,}\DataTypeTok{col=}\StringTok{'steelblue3'}\NormalTok{,}\DataTypeTok{main=}\StringTok{"Whether the subject was enrolled in }\CharTok{\textbackslash{}n}\StringTok{ PSID(==0) or NSW(==1)"}\NormalTok{,}\DataTypeTok{xlab=}\StringTok{"NSW or PSID"}\NormalTok{, }\DataTypeTok{ylab=}\StringTok{"Fraction"}\NormalTok{)}
\KeywordTok{lines}\NormalTok{(}\KeywordTok{density}\NormalTok{(trt),}\DataTypeTok{lwd=}\DecValTok{2}\NormalTok{)}

\KeywordTok{truehist}\NormalTok{(age,}\DataTypeTok{col=}\StringTok{'steelblue3'}\NormalTok{,}\DataTypeTok{main=}\StringTok{"Age"}\NormalTok{,}\DataTypeTok{xlab=}\StringTok{"age"}\NormalTok{, }\DataTypeTok{ylab=}\StringTok{"Fraction"}\NormalTok{)}
\KeywordTok{lines}\NormalTok{(}\KeywordTok{density}\NormalTok{(age),}\DataTypeTok{lwd=}\DecValTok{2}\NormalTok{)}

\KeywordTok{truehist}\NormalTok{(educ,}\DataTypeTok{col=}\StringTok{'steelblue3'}\NormalTok{,}\DataTypeTok{main=}\StringTok{"Years of Education"}\NormalTok{,}\DataTypeTok{xlab=}\StringTok{"years"}\NormalTok{, }\DataTypeTok{ylab=}\StringTok{"Fraction"}\NormalTok{)}
\KeywordTok{lines}\NormalTok{(}\KeywordTok{density}\NormalTok{(educ),}\DataTypeTok{lwd=}\DecValTok{2}\NormalTok{)}

\KeywordTok{truehist}\NormalTok{(black,}\DataTypeTok{col=}\StringTok{'steelblue3'}\NormalTok{,}\DataTypeTok{main=}\StringTok{"Whether the subject was Black"}\NormalTok{,}\DataTypeTok{xlab=}\StringTok{"black or not"}\NormalTok{, }\DataTypeTok{ylab=}\StringTok{"Fraction"}\NormalTok{)}
\KeywordTok{lines}\NormalTok{(}\KeywordTok{density}\NormalTok{(black),}\DataTypeTok{lwd=}\DecValTok{2}\NormalTok{)}
\end{Highlighting}
\end{Shaded}

\includegraphics{Homework1_files/figure-latex/unnamed-chunk-1-1.pdf}

\begin{Shaded}
\begin{Highlighting}[]
\KeywordTok{windows}\NormalTok{()}
\KeywordTok{par}\NormalTok{(}\DataTypeTok{mfrow=}\KeywordTok{c}\NormalTok{(}\DecValTok{2}\NormalTok{,}\DecValTok{2}\NormalTok{))}
\KeywordTok{truehist}\NormalTok{(hisp,}\DataTypeTok{col=}\StringTok{'steelblue3'}\NormalTok{,}\DataTypeTok{main=}\StringTok{"Whether the subject was Hispanic"}\NormalTok{,}\DataTypeTok{xlab=}\StringTok{"hispanic or not"}\NormalTok{, }\DataTypeTok{ylab=}\StringTok{"Fraction"}\NormalTok{)}
\KeywordTok{lines}\NormalTok{(}\KeywordTok{density}\NormalTok{(hisp),}\DataTypeTok{lwd=}\DecValTok{2}\NormalTok{)}

\KeywordTok{truehist}\NormalTok{(marr,}\DataTypeTok{col=}\StringTok{'steelblue3'}\NormalTok{,}\DataTypeTok{main=}\StringTok{"Whether the subject was married"}\NormalTok{,}\DataTypeTok{xlab=}\StringTok{"married or not"}\NormalTok{, }\DataTypeTok{ylab=}\StringTok{"Fraction"}\NormalTok{)}
\KeywordTok{lines}\NormalTok{(}\KeywordTok{density}\NormalTok{(marr),}\DataTypeTok{lwd=}\DecValTok{2}\NormalTok{)}

\KeywordTok{truehist}\NormalTok{(nodeg,}\DataTypeTok{col=}\StringTok{'steelblue3'}\NormalTok{,}\DataTypeTok{main=}\StringTok{"Whether the subject was a }\CharTok{\textbackslash{}n}\StringTok{ high school dropout"}\NormalTok{,}\DataTypeTok{xlab=}\StringTok{"dropout or not"}\NormalTok{, }\DataTypeTok{ylab=}\StringTok{"Fraction"}\NormalTok{)}
\KeywordTok{lines}\NormalTok{(}\KeywordTok{density}\NormalTok{(nodeg),}\DataTypeTok{lwd=}\DecValTok{2}\NormalTok{)}

\KeywordTok{truehist}\NormalTok{(re74,}\DataTypeTok{col=}\StringTok{'steelblue3'}\NormalTok{,}\DataTypeTok{main=}\StringTok{"Real Earnings in 1974"}\NormalTok{,}\DataTypeTok{xlab=}\StringTok{"real earnings"}\NormalTok{, }\DataTypeTok{ylab=}\StringTok{"Fraction"}\NormalTok{)}
\KeywordTok{lines}\NormalTok{(}\KeywordTok{density}\NormalTok{(re74),}\DataTypeTok{lwd=}\DecValTok{2}\NormalTok{)}
\end{Highlighting}
\end{Shaded}

\includegraphics{Homework1_files/figure-latex/unnamed-chunk-1-2.pdf}

\begin{Shaded}
\begin{Highlighting}[]
\KeywordTok{windows}\NormalTok{()}
\KeywordTok{par}\NormalTok{(}\DataTypeTok{mfrow=}\KeywordTok{c}\NormalTok{(}\DecValTok{1}\NormalTok{,}\DecValTok{2}\NormalTok{))}
\KeywordTok{truehist}\NormalTok{(re75,}\DataTypeTok{col=}\StringTok{'steelblue3'}\NormalTok{,}\DataTypeTok{main=}\StringTok{"Real Earnings in 1975"}\NormalTok{,}\DataTypeTok{xlab=}\StringTok{"real earnings"}\NormalTok{, }\DataTypeTok{ylab=}\StringTok{"Fraction"}\NormalTok{)}
\KeywordTok{lines}\NormalTok{(}\KeywordTok{density}\NormalTok{(re75),}\DataTypeTok{lwd=}\DecValTok{2}\NormalTok{)}

\KeywordTok{truehist}\NormalTok{(re78,}\DataTypeTok{col=}\StringTok{'steelblue3'}\NormalTok{,}\DataTypeTok{main=}\StringTok{"Real Earnings in 1978"}\NormalTok{,}\DataTypeTok{xlab=}\StringTok{"real earnings"}\NormalTok{, }\DataTypeTok{ylab=}\StringTok{"Fraction"}\NormalTok{)}
\KeywordTok{lines}\NormalTok{(}\KeywordTok{density}\NormalTok{(re78),}\DataTypeTok{lwd=}\DecValTok{2}\NormalTok{)}
\end{Highlighting}
\end{Shaded}

\includegraphics{Homework1_files/figure-latex/unnamed-chunk-1-3.pdf}

\begin{Shaded}
\begin{Highlighting}[]
\CommentTok{# Most of the data variables are binary, such as whether the subject was black or not. Even when }
\CommentTok{# the data is not binary, the pdf does not exhibit the shape of a normal distribution}

\NormalTok{##(b) Estimate your full regression model and show your results (e.g., the output from the `summary' command in R).}
\NormalTok{##    Discuss these results. }
\NormalTok{y =}\StringTok{ }\KeywordTok{lm}\NormalTok{(re78}\OperatorTok{~}\NormalTok{trt}\OperatorTok{+}\NormalTok{age}\OperatorTok{+}\NormalTok{educ}\OperatorTok{+}\NormalTok{black)}
\KeywordTok{summary}\NormalTok{(y)}
\end{Highlighting}
\end{Shaded}

\begin{verbatim}
## 
## Call:
## lm(formula = re78 ~ trt + age + educ + black)
## 
## Residuals:
##    Min     1Q Median     3Q    Max 
## -34507  -7311   -148   7071 106726 
## 
## Coefficients:
##             Estimate Std. Error t value Pr(>|t|)    
## (Intercept) -5761.73    1771.01  -3.253  0.00115 ** 
## trt         -8398.25    1156.55  -7.261 5.00e-13 ***
## age           195.55      27.29   7.165 1.00e-12 ***
## educ         1759.08      97.41  18.058  < 2e-16 ***
## black       -3247.63     668.11  -4.861 1.24e-06 ***
## ---
## Signif. codes:  0 '***' 0.001 '**' 0.01 '*' 0.05 '.' 0.1 ' ' 1
## 
## Residual standard error: 13990 on 2670 degrees of freedom
## Multiple R-squared:  0.2006, Adjusted R-squared:  0.1994 
## F-statistic: 167.5 on 4 and 2670 DF,  p-value: < 2.2e-16
\end{verbatim}

\begin{Shaded}
\begin{Highlighting}[]
\CommentTok{#Since the p-values are so small, the estimated coefficients that we obtained by running the regression}
\CommentTok{#is statistically significant. However since the R^2 is so low, the model does not fit the data}

\CommentTok{# (c) Compute the Mallows CP statistic for all the plausible models and choose only one}
\CommentTok{# model. Discuss why you chose this model. For the next questions, only use your selected}
\CommentTok{# model from this part.}
\KeywordTok{windows}\NormalTok{()}
\NormalTok{ss=leaps}\OperatorTok{::}\KeywordTok{regsubsets}\NormalTok{(re78}\OperatorTok{~}\NormalTok{trt}\OperatorTok{+}\NormalTok{age}\OperatorTok{+}\NormalTok{educ}\OperatorTok{+}\NormalTok{black,}\DataTypeTok{method=}\KeywordTok{c}\NormalTok{(}\StringTok{"exhaustive"}\NormalTok{),}\DataTypeTok{nbest=}\DecValTok{4}\NormalTok{,}\DataTypeTok{data=}\NormalTok{data)}
\KeywordTok{subsets}\NormalTok{(ss,}\DataTypeTok{statistic=}\StringTok{"cp"}\NormalTok{,}\DataTypeTok{legend=}\NormalTok{F,}\DataTypeTok{main=}\StringTok{"Mallows CP"}\NormalTok{,}\DataTypeTok{col=}\StringTok{"steelblue4"}\NormalTok{)}
\end{Highlighting}
\end{Shaded}

\begin{verbatim}
##       Abbreviation
## trt              t
## age              a
## educ             e
## black            b
\end{verbatim}

\begin{Shaded}
\begin{Highlighting}[]
\KeywordTok{legend}\NormalTok{(}\FloatTok{2.3}\NormalTok{,}\DecValTok{650}\NormalTok{,}\DataTypeTok{bty=}\StringTok{"n"}\NormalTok{,}\DataTypeTok{legend=}\KeywordTok{c}\NormalTok{(}\StringTok{'t=trt'}\NormalTok{,}\StringTok{'a=Age'}\NormalTok{,}\StringTok{'e=Years of Education'}\NormalTok{, }\StringTok{'b=Black'}\NormalTok{),}\DataTypeTok{col=}\StringTok{"steelblue4"}\NormalTok{,}\DataTypeTok{cex=}\FloatTok{1.5}\NormalTok{)}
\end{Highlighting}
\end{Shaded}

\includegraphics{Homework1_files/figure-latex/unnamed-chunk-1-4.pdf}

\begin{Shaded}
\begin{Highlighting}[]
\CommentTok{# From the Mallows Cp, we can see that the choosing all 4 variables would give the lowest Cp. }
\CommentTok{# However, the Cp is sufficiently low enough when we just take trt, age and educ variables. }
\CommentTok{# So I'll use these variables.}

\NormalTok{y1 =}\StringTok{ }\KeywordTok{lm}\NormalTok{(re78}\OperatorTok{~}\NormalTok{trt}\OperatorTok{+}\NormalTok{age}\OperatorTok{+}\NormalTok{educ)}

\CommentTok{#(d) Plot the residuals vs. the fitted values.}
\KeywordTok{windows}\NormalTok{()}
\KeywordTok{par}\NormalTok{(}\DataTypeTok{mfcol=}\KeywordTok{c}\NormalTok{(}\DecValTok{2}\NormalTok{,}\DecValTok{2}\NormalTok{))}
\KeywordTok{plot}\NormalTok{(y1}\OperatorTok{$}\NormalTok{fit,y1}\OperatorTok{$}\NormalTok{res,}\DataTypeTok{col=}\StringTok{"skyblue3"}\NormalTok{,}\DataTypeTok{pch=}\DecValTok{20}\NormalTok{,}\DataTypeTok{xlab=}\StringTok{"Predicted Response"}\NormalTok{,}\DataTypeTok{ylab=}\StringTok{"Residuals"}\NormalTok{,}\DataTypeTok{main=}\StringTok{"Residuals vs. Predicted Response"}\NormalTok{,}\DataTypeTok{cex.axis=}\FloatTok{0.8}\NormalTok{,}\DataTypeTok{cex.main=}\FloatTok{0.9}\NormalTok{)}
\KeywordTok{abline}\NormalTok{(}\DataTypeTok{h=}\DecValTok{0}\NormalTok{,}\DataTypeTok{lwd=}\DecValTok{2}\NormalTok{,}\DataTypeTok{col=}\StringTok{"red"}\NormalTok{)}
\KeywordTok{lines}\NormalTok{(}\KeywordTok{lowess}\NormalTok{(y1}\OperatorTok{$}\NormalTok{fit,y1}\OperatorTok{$}\NormalTok{res),}\DataTypeTok{lwd=}\FloatTok{1.5}\NormalTok{) }
\KeywordTok{abline}\NormalTok{(}\DataTypeTok{v=}\FloatTok{6e5}\NormalTok{,}\DataTypeTok{col=}\StringTok{"red"}\NormalTok{,}\DataTypeTok{lty=}\DecValTok{2}\NormalTok{,}\DataTypeTok{lwd=}\FloatTok{1.5}\NormalTok{)}
\KeywordTok{legend}\NormalTok{(}\OperatorTok{-}\DecValTok{5500}\NormalTok{,}\DecValTok{100000}\NormalTok{,}\KeywordTok{c}\NormalTok{(}\KeywordTok{expression}\NormalTok{(y[obs]}\OperatorTok{==}\NormalTok{y[pred]), }\StringTok{"Lowess Smoother"}\NormalTok{), }\DataTypeTok{fill =}\KeywordTok{c}\NormalTok{(}\StringTok{"red"}\NormalTok{, }\StringTok{"black"}\NormalTok{),}\DataTypeTok{cex=}\FloatTok{0.8}\NormalTok{)}

\CommentTok{#(e)Plot and discuss the VIF plot.}
\KeywordTok{plot}\NormalTok{(}\KeywordTok{vif}\NormalTok{(y1),}\DataTypeTok{col=}\StringTok{"skyblue3"}\NormalTok{,}\DataTypeTok{xlab=}\KeywordTok{rbind}\NormalTok{(}\StringTok{"trt"}\NormalTok{, }\StringTok{"age"}\NormalTok{, }\StringTok{"educ"}\NormalTok{),}\DataTypeTok{ylab=}\StringTok{"Residuals"}\NormalTok{,}\DataTypeTok{main=}\StringTok{"Residuals vs. Predicted Response"}\NormalTok{,}\DataTypeTok{cex.axis=}\FloatTok{0.8}\NormalTok{,}\DataTypeTok{cex.main=}\FloatTok{0.9}\NormalTok{)}

\CommentTok{#(f) Plot and discuss the correlation graph (use the corrplot library).}
\NormalTok{M <-}\StringTok{ }\KeywordTok{cor}\NormalTok{(data)}
\KeywordTok{corrplot}\NormalTok{(M, }\DataTypeTok{method =} \StringTok{"circle"}\NormalTok{)}

\CommentTok{#The plot shows that there is very little correlation between any 2 variables }
\CommentTok{#from the data, except that between years of education and being a high school dropout,}
\CommentTok{#and between the different years of the earning. This might imply that the other}
\CommentTok{#variables such as age, years of education, etc does not affect earnings in the population.}

\CommentTok{# (g) Plot the Cook's Distance values. Are there any outliers? If so, discuss what you would}
\CommentTok{# do with them.}
\NormalTok{y1_cook=}\KeywordTok{cooks.distance}\NormalTok{(y1)}
\KeywordTok{plot}\NormalTok{(y1_cook,}\DataTypeTok{ylab=}\StringTok{"Cook's distance"}\NormalTok{,}\DataTypeTok{type=}\StringTok{'o'}\NormalTok{,}\DataTypeTok{main=}\StringTok{"Cook's Distance Plot"}\NormalTok{,}\DataTypeTok{col=}\StringTok{"skyblue4"}\NormalTok{, }\DataTypeTok{pch=}\DecValTok{20}\NormalTok{,}\DataTypeTok{lwd=}\NormalTok{.}\DecValTok{25}\NormalTok{)}
\KeywordTok{text}\NormalTok{(}\DecValTok{7}\NormalTok{,}\FloatTok{0.28}\NormalTok{,}\StringTok{"wt=5345lbs }\CharTok{\textbackslash{}n}\StringTok{ 8-Cyl }\CharTok{\textbackslash{}n}\StringTok{  hp=230 }\CharTok{\textbackslash{}n}\StringTok{ mpg=14.7"}\NormalTok{)}
\KeywordTok{text}\NormalTok{(}\DecValTok{14}\NormalTok{,}\FloatTok{0.33}\NormalTok{, }\KeywordTok{expression}\NormalTok{(}\StringTok{""}\OperatorTok\StringTok{""}\NormalTok{))}
\end{Highlighting}
\end{Shaded}

\includegraphics{Homework1_files/figure-latex/unnamed-chunk-1-5.pdf}

\begin{Shaded}
\begin{Highlighting}[]
\CommentTok{# From the plot, we could see that there are outliers. One way to deal with it is to consider}
\CommentTok{# modifying the model, such as to a log-linear one. Since we could see that there is only one }
\CommentTok{# outlier that stands out too much, another way is to just remove that point.}

\CommentTok{#(h) Plot a histogram of the residuals and discuss the results.}
\KeywordTok{truehist}\NormalTok{(y1}\OperatorTok{$}\NormalTok{res,}\DataTypeTok{col=}\StringTok{"skyblue3"}\NormalTok{,}\DataTypeTok{xlab=}\StringTok{"Residuals"}\NormalTok{,}\DataTypeTok{ylab=}\StringTok{"Fraction"}\NormalTok{,}\DataTypeTok{main=}\StringTok{"Histogram of Residuals"}\NormalTok{)}
\NormalTok{xr=}\KeywordTok{rnorm}\NormalTok{(}\DecValTok{800000}\NormalTok{,}\KeywordTok{mean}\NormalTok{(y1}\OperatorTok{$}\NormalTok{res),}\KeywordTok{sd}\NormalTok{(y1}\OperatorTok{$}\NormalTok{res))}
\KeywordTok{lines}\NormalTok{(}\KeywordTok{density}\NormalTok{(xr),}\DataTypeTok{col=}\StringTok{"black"}\NormalTok{,}\DataTypeTok{lwd=}\DecValTok{2}\NormalTok{)}
\KeywordTok{lines}\NormalTok{(}\KeywordTok{density}\NormalTok{(y1}\OperatorTok{$}\NormalTok{res),}\DataTypeTok{col=}\StringTok{"red"}\NormalTok{,}\DataTypeTok{lwd=}\DecValTok{2}\NormalTok{)}
\KeywordTok{legend}\NormalTok{(}\FloatTok{0.3}\NormalTok{,}\FloatTok{0.2}\NormalTok{,}\KeywordTok{c}\NormalTok{(}\StringTok{"Density"}\NormalTok{,}\StringTok{"Normal Distr."}\NormalTok{),}\DataTypeTok{fill=}\KeywordTok{c}\NormalTok{(}\StringTok{"red"}\NormalTok{,}\StringTok{"black"}\NormalTok{),}\DataTypeTok{bty=}\StringTok{'n'}\NormalTok{)}

\CommentTok{# Since pdf of residuals closely resembles that of a normal distribution, we know that the OLS is unbiased.}
\CommentTok{# This means that we can estimate the coefficients of the variables without bias.}

\CommentTok{#(i) Plot the QQ Normal Plot and discuss the results.}
\KeywordTok{qqnorm}\NormalTok{(y1}\OperatorTok{$}\NormalTok{res,}\DataTypeTok{col=}\StringTok{"skyblue4"}\NormalTok{, }\DataTypeTok{pch=}\DecValTok{20}\NormalTok{,}\DataTypeTok{lwd=}\DecValTok{1}\NormalTok{,}\DataTypeTok{main=}\StringTok{"QQ Normal Plot"}\NormalTok{)}

\CommentTok{#change this}
\CommentTok{# Since the shape of the line is fairly straight, we can assume that linear regression works fine.}
\CommentTok{# However we could that there are some points that jump at the right tail, which implies that we have outliers.}

\CommentTok{#(j) Plot the observed vs. predicted values, overlay a Lowess smoother, and discuss the}
\CommentTok{#results.}
\KeywordTok{windows}\NormalTok{()}
\KeywordTok{plot}\NormalTok{(y1}\OperatorTok{$}\NormalTok{fit,re78,}\DataTypeTok{pch=}\DecValTok{20}\NormalTok{,}\DataTypeTok{col=}\StringTok{"skyblue4"}\NormalTok{,}\DataTypeTok{cex=}\DecValTok{1}\NormalTok{,}\DataTypeTok{xlab=}\StringTok{"Predicted Response"}\NormalTok{,}\DataTypeTok{ylab=}\StringTok{"Observed Response"}\NormalTok{,}\DataTypeTok{main=}\StringTok{"Observed vs. Predicted Response }\CharTok{\textbackslash{}n}\StringTok{ Full Model"}\NormalTok{,}\DataTypeTok{cex.axis=}\FloatTok{0.8}\NormalTok{,}\DataTypeTok{cex.main=}\FloatTok{1.0}\NormalTok{)}
\KeywordTok{lines}\NormalTok{(}\KeywordTok{lowess}\NormalTok{(y1}\OperatorTok{$}\NormalTok{fit,re78),}\DataTypeTok{lwd=}\DecValTok{2}\NormalTok{)}
\KeywordTok{abline}\NormalTok{(}\DecValTok{0}\NormalTok{,}\DecValTok{1}\NormalTok{,}\DataTypeTok{col=}\StringTok{"red"}\NormalTok{,}\DataTypeTok{lwd=}\DecValTok{2}\NormalTok{,}\DataTypeTok{lty=}\DecValTok{2}\NormalTok{) }
\KeywordTok{text}\NormalTok{(}\DecValTok{25}\NormalTok{,}\DecValTok{18}\NormalTok{,}\KeywordTok{expression}\NormalTok{(R}\OperatorTok{^}\DecValTok{2}\OperatorTok{==}\FloatTok{0.826}\NormalTok{))}
\KeywordTok{legend}\NormalTok{(}\DecValTok{15}\NormalTok{,}\DecValTok{31}\NormalTok{, }\KeywordTok{c}\NormalTok{(}\KeywordTok{expression}\NormalTok{(y[obs]}\OperatorTok{==}\NormalTok{y[pred]), }\StringTok{"Lowess Smoother"}\NormalTok{), }\DataTypeTok{fill =}\KeywordTok{c}\NormalTok{(}\StringTok{"red"}\NormalTok{, }\StringTok{"black"}\NormalTok{),}\DataTypeTok{cex=}\DecValTok{1}\NormalTok{,}\DataTypeTok{bty=}\StringTok{"y"}\NormalTok{)}
\end{Highlighting}
\end{Shaded}

\includegraphics{Homework1_files/figure-latex/unnamed-chunk-1-6.pdf}

\begin{Shaded}
\begin{Highlighting}[]
\CommentTok{# The way the plots scatter all over indicates that there is very little correspondence }
\CommentTok{# between the observed values and the predicted values. This means that the model for prediction  }
\CommentTok{# is poor}
\end{Highlighting}
\end{Shaded}

\begin{Shaded}
\begin{Highlighting}[]
\CommentTok{#2}
\CommentTok{#clear all variables and prior sessions}
\KeywordTok{rm}\NormalTok{(}\DataTypeTok{list=}\KeywordTok{ls}\NormalTok{(}\DataTypeTok{all=}\OtherTok{TRUE}\NormalTok{))}

\CommentTok{#get the data and attach it}
\NormalTok{data =}\StringTok{ }\KeywordTok{read.csv}\NormalTok{(}\StringTok{"2.2.csv"}\NormalTok{, }\DataTypeTok{header =}\NormalTok{ T)}
\KeywordTok{attach}\NormalTok{(data)}

\CommentTok{#obtain the descriptive statistics: mean, median, variance, standard deviation,}
\CommentTok{#skewness, and kurtosis.}
\KeywordTok{print}\NormalTok{(}\KeywordTok{paste}\NormalTok{(}\StringTok{"The mean of the GDP quarterly growth rates is "}\NormalTok{, }\KeywordTok{mean}\NormalTok{(GRGDP)))}
\end{Highlighting}
\end{Shaded}

\begin{verbatim}
## [1] "The mean of the GDP quarterly growth rates is  0.795803870967742"
\end{verbatim}

\begin{Shaded}
\begin{Highlighting}[]
\KeywordTok{print}\NormalTok{(}\KeywordTok{paste}\NormalTok{(}\StringTok{"The median of the GDP quarterly growth rates is "}\NormalTok{, }\KeywordTok{median}\NormalTok{(GRGDP)))}
\end{Highlighting}
\end{Shaded}

\begin{verbatim}
## [1] "The median of the GDP quarterly growth rates is  0.76789"
\end{verbatim}

\begin{Shaded}
\begin{Highlighting}[]
\KeywordTok{print}\NormalTok{(}\KeywordTok{paste}\NormalTok{(}\StringTok{"The variance of the GDP quarterly growth rates is "}\NormalTok{, }\KeywordTok{var}\NormalTok{(GRGDP)))}
\end{Highlighting}
\end{Shaded}

\begin{verbatim}
## [1] "The variance of the GDP quarterly growth rates is  0.951862567096696"
\end{verbatim}

\begin{Shaded}
\begin{Highlighting}[]
\KeywordTok{print}\NormalTok{(}\KeywordTok{paste}\NormalTok{(}\StringTok{"The standard deviation of the GDP quarterly growth rates is "}\NormalTok{, (}\KeywordTok{var}\NormalTok{(GRGDP)}\OperatorTok{^}\NormalTok{(}\FloatTok{0.5}\NormalTok{))))}
\end{Highlighting}
\end{Shaded}

\begin{verbatim}
## [1] "The standard deviation of the GDP quarterly growth rates is  0.975634443373488"
\end{verbatim}

\begin{Shaded}
\begin{Highlighting}[]
\KeywordTok{print}\NormalTok{(}\KeywordTok{paste}\NormalTok{(}\StringTok{"The skewness of the GDP quarterly growth rates is "}\NormalTok{, }\KeywordTok{skewness}\NormalTok{(GRGDP)))}
\end{Highlighting}
\end{Shaded}

\begin{verbatim}
## [1] "The skewness of the GDP quarterly growth rates is  -0.187714593795853"
\end{verbatim}

\begin{Shaded}
\begin{Highlighting}[]
\KeywordTok{print}\NormalTok{(}\KeywordTok{paste}\NormalTok{(}\StringTok{"The kurtosis of the GDP quarterly growth rates is "}\NormalTok{, }\KeywordTok{kurtosis}\NormalTok{(GRGDP)))}
\end{Highlighting}
\end{Shaded}

\begin{verbatim}
## [1] "The kurtosis of the GDP quarterly growth rates is  1.38288523763476"
\end{verbatim}

\begin{Shaded}
\begin{Highlighting}[]
\KeywordTok{print}\NormalTok{(}\KeywordTok{paste}\NormalTok{(}\StringTok{"The mean of the S&P500 quarterly returns is "}\NormalTok{, }\KeywordTok{mean}\NormalTok{(RETURN)))}
\end{Highlighting}
\end{Shaded}

\begin{verbatim}
## [1] "The mean of the S&P500 quarterly returns is  1.96210425806452"
\end{verbatim}

\begin{Shaded}
\begin{Highlighting}[]
\KeywordTok{print}\NormalTok{(}\KeywordTok{paste}\NormalTok{(}\StringTok{"The median of the S&P500 quarterly returns is "}\NormalTok{, }\KeywordTok{median}\NormalTok{(RETURN)))}
\end{Highlighting}
\end{Shaded}

\begin{verbatim}
## [1] "The median of the S&P500 quarterly returns is  2.0227015"
\end{verbatim}

\begin{Shaded}
\begin{Highlighting}[]
\KeywordTok{print}\NormalTok{(}\KeywordTok{paste}\NormalTok{(}\StringTok{"The variance of the S&P500 quarterly returns is "}\NormalTok{, }\KeywordTok{var}\NormalTok{(RETURN)))}
\end{Highlighting}
\end{Shaded}

\begin{verbatim}
## [1] "The variance of the S&P500 quarterly returns is  36.9271348635885"
\end{verbatim}

\begin{Shaded}
\begin{Highlighting}[]
\KeywordTok{print}\NormalTok{(}\KeywordTok{paste}\NormalTok{(}\StringTok{"The standard deviation of the S&P500 quarterly returns is "}\NormalTok{, (}\KeywordTok{var}\NormalTok{(RETURN)}\OperatorTok{^}\NormalTok{(}\FloatTok{0.5}\NormalTok{))))}
\end{Highlighting}
\end{Shaded}

\begin{verbatim}
## [1] "The standard deviation of the S&P500 quarterly returns is  6.07677010126173"
\end{verbatim}

\begin{Shaded}
\begin{Highlighting}[]
\KeywordTok{print}\NormalTok{(}\KeywordTok{paste}\NormalTok{(}\StringTok{"The skewness of the S&P500 quarterly returns is "}\NormalTok{, }\KeywordTok{skewness}\NormalTok{(RETURN)))}
\end{Highlighting}
\end{Shaded}

\begin{verbatim}
## [1] "The skewness of the S&P500 quarterly returns is  -0.679887149724607"
\end{verbatim}

\begin{Shaded}
\begin{Highlighting}[]
\KeywordTok{print}\NormalTok{(}\KeywordTok{paste}\NormalTok{(}\StringTok{"The kurtosis of the S&P500 quarterly returns is "}\NormalTok{, }\KeywordTok{kurtosis}\NormalTok{(RETURN)))}
\end{Highlighting}
\end{Shaded}

\begin{verbatim}
## [1] "The kurtosis of the S&P500 quarterly returns is  2.22845965467761"
\end{verbatim}

\begin{Shaded}
\begin{Highlighting}[]
\CommentTok{#histograms}
\KeywordTok{windows}\NormalTok{()}
\KeywordTok{par}\NormalTok{(}\DataTypeTok{mfrow=}\KeywordTok{c}\NormalTok{(}\DecValTok{1}\NormalTok{,}\DecValTok{2}\NormalTok{))}
\KeywordTok{truehist}\NormalTok{(GRGDP,}\DataTypeTok{col=}\StringTok{'steelblue3'}\NormalTok{,}\DataTypeTok{main=}\StringTok{"U.S. GDP quarterly growth rates"}\NormalTok{,}\DataTypeTok{xlab=}\StringTok{"Rates"}\NormalTok{, }\DataTypeTok{ylab=}\StringTok{"Fraction"}\NormalTok{,}
         \DataTypeTok{xlim=}\KeywordTok{c}\NormalTok{(}\OperatorTok{-}\DecValTok{5}\NormalTok{,}\DecValTok{5}\NormalTok{),}\DataTypeTok{ylim=}\KeywordTok{c}\NormalTok{(}\DecValTok{0}\NormalTok{,}\FloatTok{0.7}\NormalTok{))}
\KeywordTok{lines}\NormalTok{(}\KeywordTok{density}\NormalTok{(GRGDP),}\DataTypeTok{lwd=}\DecValTok{2}\NormalTok{)}

\KeywordTok{truehist}\NormalTok{(RETURN,}\DataTypeTok{col=}\StringTok{'steelblue3'}\NormalTok{,}\DataTypeTok{main=}\StringTok{"S&P 500 quarterly returns"}\NormalTok{,}\DataTypeTok{xlab=}\StringTok{"Returns"}\NormalTok{, }\DataTypeTok{ylab=}\StringTok{"Fraction"}\NormalTok{,}
         \DataTypeTok{xlim=}\KeywordTok{c}\NormalTok{(}\OperatorTok{-}\DecValTok{40}\NormalTok{,}\DecValTok{50}\NormalTok{),}\DataTypeTok{ylim=}\KeywordTok{c}\NormalTok{(}\DecValTok{0}\NormalTok{,}\FloatTok{0.09}\NormalTok{))}
\KeywordTok{lines}\NormalTok{(}\KeywordTok{density}\NormalTok{(RETURN),}\DataTypeTok{lwd=}\DecValTok{2}\NormalTok{)}
\end{Highlighting}
\end{Shaded}

\includegraphics{Homework1_files/figure-latex/unnamed-chunk-2-1.pdf}

\begin{Shaded}
\begin{Highlighting}[]
\CommentTok{#correlation }
\KeywordTok{print}\NormalTok{ (}\StringTok{"The correlation matrix is: "}\NormalTok{)}
\end{Highlighting}
\end{Shaded}

\begin{verbatim}
## [1] "The correlation matrix is: "
\end{verbatim}

\begin{Shaded}
\begin{Highlighting}[]
\KeywordTok{cor}\NormalTok{(}\KeywordTok{cbind}\NormalTok{(GRGDP, RETURN))}
\end{Highlighting}
\end{Shaded}

\begin{verbatim}
##            GRGDP    RETURN
## GRGDP  1.0000000 0.2702427
## RETURN 0.2702427 1.0000000
\end{verbatim}

\begin{Shaded}
\begin{Highlighting}[]
\CommentTok{#This statistics shows that the median is higher slightly compared to the mean,}
\CommentTok{#the data is skewed slightly to the right, which can be verified by looking at the }
\CommentTok{#negative skewness and the shape of the histogram. }
\CommentTok{#The correlation between US GDP growth rate and SP500 return is also very low.}

\CommentTok{#3.}
\CommentTok{#clear all variables and prior sessions and obtain the data}
\KeywordTok{rm}\NormalTok{(}\DataTypeTok{list=}\KeywordTok{ls}\NormalTok{(}\DataTypeTok{all=}\OtherTok{TRUE}\NormalTok{))}
\NormalTok{data =}\StringTok{ }\KeywordTok{read.csv}\NormalTok{(}\StringTok{"3.1.csv"}\NormalTok{, }\DataTypeTok{header =}\NormalTok{ T)}
\KeywordTok{attach}\NormalTok{(data)}

\CommentTok{#a)}
\NormalTok{rpce_pctchg =}\StringTok{ }\KeywordTok{diff}\NormalTok{(}\KeywordTok{log}\NormalTok{(rpce))}
\NormalTok{rdpi_pctchg =}\StringTok{ }\KeywordTok{diff}\NormalTok{(}\KeywordTok{log}\NormalTok{(rdpi))}

\KeywordTok{windows}\NormalTok{()}
\KeywordTok{par}\NormalTok{(}\DataTypeTok{mfrow =} \KeywordTok{c}\NormalTok{(}\DecValTok{2}\NormalTok{,}\DecValTok{1}\NormalTok{))}
\KeywordTok{plot}\NormalTok{(rpce_pctchg, }\DataTypeTok{main=}\StringTok{"Scatterplot of real personal consumption expenditure growth rate"}\NormalTok{, }
     \DataTypeTok{ylab=}\StringTok{"real personal consumption expenditure growth rate"}\NormalTok{, }\DataTypeTok{pch=}\DecValTok{19}\NormalTok{)}
\KeywordTok{plot}\NormalTok{(rdpi_pctchg, }\DataTypeTok{main=}\StringTok{"Scatterplot of real disposal personal income"}\NormalTok{,  }
     \DataTypeTok{ylab=}\StringTok{"real disposal personal income"}\NormalTok{, }\DataTypeTok{pch=}\DecValTok{19}\NormalTok{)}
\end{Highlighting}
\end{Shaded}

\includegraphics{Homework1_files/figure-latex/unnamed-chunk-2-2.pdf}

\begin{Shaded}
\begin{Highlighting}[]
\CommentTok{#change this}
\CommentTok{#from the plots, we can see that growth rate of the real personal consumption expenditure}
\CommentTok{#growth rate is more volatile than that of real disposal personal income}
\CommentTok{#Permanent income model that the current expenditure is not determined just by their }
\CommentTok{#current income but also by their future expected income. As such, it would mean }
\CommentTok{#that even if a person's income stays relatively constant, his consumption might vary alot }
\CommentTok{#depending on whether he wants to spend less in the current time and spend more later or}
\CommentTok{#vice versa. This would mean that that the model explains the volatility difference.}

\CommentTok{#b)}
\NormalTok{y =}\StringTok{ }\KeywordTok{lm}\NormalTok{(rpce_pctchg}\OperatorTok{~}\NormalTok{rdpi_pctchg)}
\KeywordTok{summary}\NormalTok{(y)}
\end{Highlighting}
\end{Shaded}

\begin{verbatim}
## 
## Call:
## lm(formula = rpce_pctchg ~ rdpi_pctchg)
## 
## Residuals:
##        Min         1Q     Median         3Q        Max 
## -0.0303967 -0.0029727  0.0001525  0.0030417  0.0244545 
## 
## Coefficients:
##              Estimate Std. Error t value Pr(>|t|)    
## (Intercept) 0.0022542  0.0002242  10.055  < 2e-16 ***
## rdpi_pctchg 0.1745671  0.0292029   5.978 3.77e-09 ***
## ---
## Signif. codes:  0 '***' 0.001 '**' 0.01 '*' 0.05 '.' 0.1 ' ' 1
## 
## Residual standard error: 0.005318 on 637 degrees of freedom
## Multiple R-squared:  0.05312,    Adjusted R-squared:  0.05163 
## F-statistic: 35.73 on 1 and 637 DF,  p-value: 3.768e-09
\end{verbatim}

\begin{Shaded}
\begin{Highlighting}[]
\CommentTok{#both values for the intercept and the coefficient of income have very low p value,}
\CommentTok{#which means that they are statistically significant.}
\CommentTok{#the intercept is close to 0, which means that if the person's income is 0, he would not consume. }
\CommentTok{#The income coefficient is positive, which means that the higher the income,}
\CommentTok{#the more the person will consume}

\CommentTok{#c)}
\NormalTok{y =}\StringTok{ }\KeywordTok{lm}\NormalTok{(rpce_pctchg}\OperatorTok{~}\NormalTok{rdpi_pctchg}\OperatorTok{+}\KeywordTok{lag}\NormalTok{(rdpi_pctchg))}
\KeywordTok{summary}\NormalTok{(y)}
\end{Highlighting}
\end{Shaded}

\begin{verbatim}
## 
## Call:
## lm(formula = rpce_pctchg ~ rdpi_pctchg + lag(rdpi_pctchg))
## 
## Residuals:
##        Min         1Q     Median         3Q        Max 
## -0.0303967 -0.0029727  0.0001525  0.0030417  0.0244545 
## 
## Coefficients: (1 not defined because of singularities)
##                   Estimate Std. Error t value Pr(>|t|)    
## (Intercept)      0.0022542  0.0002242  10.055  < 2e-16 ***
## rdpi_pctchg      0.1745671  0.0292029   5.978 3.77e-09 ***
## lag(rdpi_pctchg)        NA         NA      NA       NA    
## ---
## Signif. codes:  0 '***' 0.001 '**' 0.01 '*' 0.05 '.' 0.1 ' ' 1
## 
## Residual standard error: 0.005318 on 637 degrees of freedom
## Multiple R-squared:  0.05312,    Adjusted R-squared:  0.05163 
## F-statistic: 35.73 on 1 and 637 DF,  p-value: 3.768e-09
\end{verbatim}

\begin{Shaded}
\begin{Highlighting}[]
\CommentTok{#4.}
\CommentTok{#clear all variables and prior sessions and obtain the data}
\KeywordTok{rm}\NormalTok{(}\DataTypeTok{list=}\KeywordTok{ls}\NormalTok{(}\DataTypeTok{all=}\OtherTok{TRUE}\NormalTok{))}
\NormalTok{rgdp1 =}\StringTok{ }\KeywordTok{read.csv}\NormalTok{(}\StringTok{"3.3a.csv"}\NormalTok{, }\DataTypeTok{header =}\NormalTok{ T)}
\KeywordTok{attach}\NormalTok{(rgdp1)}
\end{Highlighting}
\end{Shaded}

\begin{verbatim}
## The following object is masked from data (pos = 3):
## 
##     X
\end{verbatim}

\begin{verbatim}
## The following object is masked from data (pos = 4):
## 
##     date
\end{verbatim}

\begin{Shaded}
\begin{Highlighting}[]
\NormalTok{rgdp_ts<-}\KeywordTok{ts}\NormalTok{(rgdp,}\DataTypeTok{start=}\DecValTok{1948}\NormalTok{,}\DataTypeTok{freq=}\DecValTok{4}\NormalTok{)}

\NormalTok{jpusforex =}\StringTok{ }\KeywordTok{read.csv}\NormalTok{(}\StringTok{"3.3b.csv"}\NormalTok{, }\DataTypeTok{header =}\NormalTok{ T)}
\KeywordTok{attach}\NormalTok{(jpusforex)}
\end{Highlighting}
\end{Shaded}

\begin{verbatim}
## The following object is masked from rgdp1:
## 
##     X
\end{verbatim}

\begin{verbatim}
## The following object is masked from data (pos = 4):
## 
##     X
\end{verbatim}

\begin{Shaded}
\begin{Highlighting}[]
\NormalTok{jpusforex_ts<-}\KeywordTok{ts}\NormalTok{(jpy_usd,}\DataTypeTok{start=}\DecValTok{1971}\NormalTok{,}\DataTypeTok{freq=}\DecValTok{252}\NormalTok{)}

\NormalTok{tcmr =}\StringTok{ }\KeywordTok{read.csv}\NormalTok{(}\StringTok{"3.3c.csv"}\NormalTok{, }\DataTypeTok{header =}\NormalTok{ T)}
\KeywordTok{attach}\NormalTok{(tcmr)}
\end{Highlighting}
\end{Shaded}

\begin{verbatim}
## The following objects are masked from jpusforex:
## 
##     DATE, X, X.1, X.2
\end{verbatim}

\begin{verbatim}
## The following object is masked from rgdp1:
## 
##     X
\end{verbatim}

\begin{verbatim}
## The following object is masked from data (pos = 5):
## 
##     X
\end{verbatim}

\begin{Shaded}
\begin{Highlighting}[]
\NormalTok{tcmr_ts<-}\KeywordTok{ts}\NormalTok{(CMRate10Yr,}\DataTypeTok{start=}\DecValTok{1971}\NormalTok{,}\DataTypeTok{freq=}\DecValTok{252}\NormalTok{)}

\NormalTok{cer =}\StringTok{ }\KeywordTok{read.csv}\NormalTok{(}\StringTok{"3.3d.csv"}\NormalTok{, }\DataTypeTok{header =}\NormalTok{ T)}
\KeywordTok{attach}\NormalTok{(cer)}
\end{Highlighting}
\end{Shaded}

\begin{verbatim}
## The following objects are masked from tcmr:
## 
##     DATE, X, X.1, X.2
\end{verbatim}

\begin{verbatim}
## The following objects are masked from jpusforex:
## 
##     DATE, X, X.1, X.2
\end{verbatim}

\begin{verbatim}
## The following object is masked from rgdp1:
## 
##     X
\end{verbatim}

\begin{verbatim}
## The following object is masked from data (pos = 6):
## 
##     X
\end{verbatim}

\begin{Shaded}
\begin{Highlighting}[]
\NormalTok{cer_ts<-}\KeywordTok{ts}\NormalTok{(unemrate,}\DataTypeTok{start=}\DecValTok{1948}\NormalTok{, }\DataTypeTok{freq=}\DecValTok{12}\NormalTok{)}

\KeywordTok{windows}\NormalTok{()}
\KeywordTok{par}\NormalTok{(}\DataTypeTok{mfrow =} \KeywordTok{c}\NormalTok{(}\DecValTok{1}\NormalTok{,}\DecValTok{2}\NormalTok{))}
\KeywordTok{plot}\NormalTok{(rgdp_ts, }\DataTypeTok{main=}\StringTok{"Scatterplot of real GDP"}\NormalTok{,  }
     \DataTypeTok{ylab=}\StringTok{"real GDP"}\NormalTok{, }\DataTypeTok{pch=}\DecValTok{19}\NormalTok{)}
\KeywordTok{plot}\NormalTok{(jpusforex_ts, }\DataTypeTok{main=}\StringTok{"Scatterplot of Japan-US FOREX"}\NormalTok{,  }
     \DataTypeTok{ylab=}\StringTok{"Jp-US Forex"}\NormalTok{, }\DataTypeTok{pch=}\DecValTok{19}\NormalTok{)}
\end{Highlighting}
\end{Shaded}

\includegraphics{Homework1_files/figure-latex/unnamed-chunk-2-3.pdf}

\begin{Shaded}
\begin{Highlighting}[]
\KeywordTok{windows}\NormalTok{()}
\KeywordTok{par}\NormalTok{(}\DataTypeTok{mfrow =} \KeywordTok{c}\NormalTok{(}\DecValTok{1}\NormalTok{,}\DecValTok{2}\NormalTok{))}
\KeywordTok{plot}\NormalTok{(tcmr_ts, }\DataTypeTok{main=}\StringTok{"Scatterplot of Treasure }\CharTok{\textbackslash{}n}\StringTok{ Constant Maturity Rate"}\NormalTok{,  }
     \DataTypeTok{ylab=}\StringTok{"tcmr"}\NormalTok{, }\DataTypeTok{pch=}\DecValTok{19}\NormalTok{)}
\KeywordTok{plot}\NormalTok{(cer_ts, }\DataTypeTok{main=}\StringTok{"Scatterplot of }\CharTok{\textbackslash{}n}\StringTok{ Civilian Unemployment Rate"}\NormalTok{,  }
     \DataTypeTok{ylab=}\StringTok{"cur"}\NormalTok{, }\DataTypeTok{pch=}\DecValTok{19}\NormalTok{)}
\end{Highlighting}
\end{Shaded}

\includegraphics{Homework1_files/figure-latex/unnamed-chunk-2-4.pdf}


\end{document}
